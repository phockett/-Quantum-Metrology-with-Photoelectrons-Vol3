%% Generated by Sphinx.
\def\sphinxdocclass{jupyterBook}
\documentclass[letterpaper,10pt,english]{jupyterBook}
\ifdefined\pdfpxdimen
   \let\sphinxpxdimen\pdfpxdimen\else\newdimen\sphinxpxdimen
\fi \sphinxpxdimen=.75bp\relax
\ifdefined\pdfimageresolution
    \pdfimageresolution= \numexpr \dimexpr1in\relax/\sphinxpxdimen\relax
\fi
%% let collapsible pdf bookmarks panel have high depth per default
\PassOptionsToPackage{bookmarksdepth=5}{hyperref}
%% turn off hyperref patch of \index as sphinx.xdy xindy module takes care of
%% suitable \hyperpage mark-up, working around hyperref-xindy incompatibility
\PassOptionsToPackage{hyperindex=false}{hyperref}
%% memoir class requires extra handling
\makeatletter\@ifclassloaded{memoir}
{\ifdefined\memhyperindexfalse\memhyperindexfalse\fi}{}\makeatother

\PassOptionsToPackage{warn}{textcomp}

\catcode`^^^^00a0\active\protected\def^^^^00a0{\leavevmode\nobreak\ }
\usepackage{cmap}
\usepackage{fontspec}
\defaultfontfeatures[\rmfamily,\sffamily,\ttfamily]{}
\usepackage{amsmath,amssymb,amstext}
\usepackage{polyglossia}
\setmainlanguage{english}



\setmainfont{FreeSerif}[
  Extension      = .otf,
  UprightFont    = *,
  ItalicFont     = *Italic,
  BoldFont       = *Bold,
  BoldItalicFont = *BoldItalic
]
\setsansfont{FreeSans}[
  Extension      = .otf,
  UprightFont    = *,
  ItalicFont     = *Oblique,
  BoldFont       = *Bold,
  BoldItalicFont = *BoldOblique,
]
\setmonofont{FreeMono}[
  Extension      = .otf,
  UprightFont    = *,
  ItalicFont     = *Oblique,
  BoldFont       = *Bold,
  BoldItalicFont = *BoldOblique,
]



\usepackage[Bjarne]{fncychap}
\usepackage[,numfigreset=1,mathnumfig]{sphinx}

\fvset{fontsize=\small}
\usepackage{geometry}


% Include hyperref last.
\usepackage{hyperref}
% Fix anchor placement for figures with captions.
\usepackage{hypcap}% it must be loaded after hyperref.
% Set up styles of URL: it should be placed after hyperref.
\urlstyle{same}

\addto\captionsenglish{\renewcommand{\contentsname}{Introdution to the tools}}

\usepackage{sphinxmessages}



        % Start of preamble defined in sphinx-jupyterbook-latex %
         \usepackage[Latin,Greek]{ucharclasses}
        \usepackage{unicode-math}
        % fixing title of the toc
        \addto\captionsenglish{\renewcommand{\contentsname}{Contents}}
        \hypersetup{
            pdfencoding=auto,
            psdextra
        }
        % End of preamble defined in sphinx-jupyterbook-latex %
        

\title{Quantum Metrology with Photoelectrons Vol. 3}
\date{Jan 26, 2022}
\release{}
\author{Paul Hockett}
\newcommand{\sphinxlogo}{\vbox{}}
\renewcommand{\releasename}{}
\makeindex
\begin{document}

\pagestyle{empty}
\sphinxmaketitle
\pagestyle{plain}
\sphinxtableofcontents
\pagestyle{normal}
\phantomsection\label{\detokenize{intro::doc}}


\sphinxAtStartPar
Quantum metrology with photoelectrons volume 3, open source executable book. This repository contains the source documents (mainly Jupyter Notebooks in Python) and notes for the book, as of Jan 2022 writing is in progress, and the \sphinxhref{https://phockett.github.io/Quantum-Metrology-with-Photoelectrons-Vol3/}{current HTML build can be found online}. The book is due to be finished in 2023, and will be published by IOP Press \sphinxhyphen{} see below for more details.

\begin{DUlineblock}{0em}
\item[] \sphinxstylestrong{\Large Series abstract}
\end{DUlineblock}

\sphinxAtStartPar
Photoionization is an interferometric process, in which multiple paths can contribute to the final continuum photoelectron wavefunction. At the simplest level, interferences between different final angular momentum states are manifest in the energy and angle resolved photoelectron spectra: metrology schemes making use of these interferograms are thus phase\sphinxhyphen{}sensitive, and provide a powerful route to detailed understanding of photoionization. In these cases, the continuum wavefunction (and underlying scattering dynamics) can be characterised. At a more complex level, such measurements can also provide a powerful probe for other processes of interest, leading to a more general class of quantum metrology built on phase\sphinxhyphen{}sensitive photoelectron imaging.  Since the turn of the century, the increasing availability of photoelectron imaging experiments, along with the increasing sophistication of experimental techniques, and the availability of computational resources for analysis and numerics, has allowed for significant developments in such photoelectron metrology.

\begin{DUlineblock}{0em}
\item[] \sphinxstylestrong{\large About the books}
\end{DUlineblock}

\sphinxAtStartPar
\sphinxincludegraphics{{mock_covers_2vol_020318}.png}
\begin{itemize}
\item {} 
\sphinxAtStartPar
Volume I covers the core physics of photoionization, including a range of computational examples. The material is presented as both reference and tutorial, and should appeal to readers of all levels. ISBN 978\sphinxhyphen{}1\sphinxhyphen{}6817\sphinxhyphen{}4684\sphinxhyphen{}5, \sphinxurl{http://iopscience.iop.org/book/978-1-6817-4684-5} (IOP Press, 2018)

\item {} 
\sphinxAtStartPar
Volume II explores applications, and the development of quantum metrology schemes based on photoelectron measurements. The material is more technical, and will appeal more to the specialist reader. ISBN 978\sphinxhyphen{}1\sphinxhyphen{}6817\sphinxhyphen{}4688\sphinxhyphen{}3, \sphinxurl{http://iopscience.iop.org/book/978-1-6817-4688-3} (IOP Press, 2018)

\end{itemize}

\sphinxAtStartPar
Additional online resources for Vols. I \& II can be found on \sphinxhref{https://osf.io/q2v3g/wiki/home/}{OSF} and \sphinxhref{https://github.com/phockett/Quantum-Metrology-with-Photoelectrons}{Github}.
\begin{itemize}
\item {} 
\sphinxAtStartPar
Volume III in the series will continue this exploration, with a focus on numerical analysis techniques, forging a closer link between experimental and theoretical results, and making the methodologies discussed directly accessible via new software. The book is due for publication by IOP due in 2023; this volume is also open\sphinxhyphen{}source, with a live HTML version at \sphinxurl{https://phockett.github.io/Quantum-Metrology-with-Photoelectrons-Vol3/} and source available at \sphinxurl{https://github.com/phockett/Quantum-Metrology-with-Photoelectrons-Vol3}.

\end{itemize}

\sphinxAtStartPar
For some additional details and motivations (including topical video), see \sphinxhref{https://phockett.github.io/ePSdata/about.html\#Motivation}{the ePSdata project}.

\begin{DUlineblock}{0em}
\item[] \sphinxstylestrong{\large Technical details}
\end{DUlineblock}

\sphinxAtStartPar
This repository contains:
\begin{itemize}
\item {} 
\sphinxAtStartPar
\sphinxcode{\sphinxupquote{doc\sphinxhyphen{}source}}: the source documents (mainly Jupyter Notebooks in Python)

\item {} 
\sphinxAtStartPar
\sphinxcode{\sphinxupquote{notes}}: additional notes for the book,

\item {} 
\sphinxAtStartPar
the \sphinxcode{\sphinxupquote{gh\sphinxhyphen{}pages}} branch contains the current HTML build, also available at \sphinxurl{https://phockett.github.io/Quantum-Metrology-with-Photoelectrons-Vol3/}

\end{itemize}

\sphinxAtStartPar
The project has been setup to use the \sphinxhref{https://jupyterbook.org/}{Jupyter Book} build\sphinxhyphen{}chain (which uses Sphinx on the back\sphinxhyphen{}end) to generate HTML and Latex outputs for publication from source Jupyter notebooks \& markdown files.

\sphinxAtStartPar
The work \sphinxstyleemphasis{within} the book will make use of the \sphinxhref{https://pemtk.readthedocs.io/en/latest/about.html}{Photoelectron Metrology Toolkit} platform for working with experimental \& theoretical data.

\sphinxAtStartPar
\sphinxincludegraphics{{ccdcf3feb912fb992ab79da89d86a2521bfe1c21}.png}

\begin{DUlineblock}{0em}
\item[] \sphinxstylestrong{\large Running code examples}
\end{DUlineblock}

\sphinxAtStartPar
Each Jupyter notebook (\sphinxcode{\sphinxupquote{*.ipynb}}) can be treated as a stand\sphinxhyphen{}alone computational document. These can be run/used/modified independently with an appropriately setup python environment (details to follow).

\begin{DUlineblock}{0em}
\item[] \sphinxstylestrong{\large Building the book}
\end{DUlineblock}

\sphinxAtStartPar
The full book can also be built from source:
\begin{enumerate}
\sphinxsetlistlabels{\arabic}{enumi}{enumii}{}{.}%
\item {} 
\sphinxAtStartPar
Clone this repository

\item {} 
\sphinxAtStartPar
Run \sphinxcode{\sphinxupquote{pip install \sphinxhyphen{}r requirements.txt}} (it is recommended you do this within a virtual environment)

\item {} 
\sphinxAtStartPar
(Optional) Edit the books source files located in the \sphinxcode{\sphinxupquote{doc\sphinxhyphen{}source/}} directory

\item {} 
\sphinxAtStartPar
Run \sphinxcode{\sphinxupquote{jupyter\sphinxhyphen{}book clean doc\sphinxhyphen{}source/}} to remove any existing builds

\item {} 
\sphinxAtStartPar
For an HTML build:
\begin{itemize}
\item {} 
\sphinxAtStartPar
Run \sphinxcode{\sphinxupquote{jupyter\sphinxhyphen{}book build doc\sphinxhyphen{}source/}}

\item {} 
\sphinxAtStartPar
A fully\sphinxhyphen{}rendered HTML version of the book will be built in \sphinxcode{\sphinxupquote{doc\sphinxhyphen{}source/\_build/html/}}.

\end{itemize}

\item {} 
\sphinxAtStartPar
For a LaTex \& PDF build:
\begin{itemize}
\item {} 
\sphinxAtStartPar
Run \sphinxcode{\sphinxupquote{jupyter\sphinxhyphen{}book build doc\sphinxhyphen{}source/ \sphinxhyphen{}\sphinxhyphen{}builder pdflatex}}

\item {} 
\sphinxAtStartPar
A fully\sphinxhyphen{}rendered HTML version of the book will be built in \sphinxcode{\sphinxupquote{doc\sphinxhyphen{}source/\_build/latex/}}.

\end{itemize}

\end{enumerate}

\sphinxAtStartPar
See \sphinxurl{https://jupyterbook.org/basics/building/index.html} for more information.

\begin{DUlineblock}{0em}
\item[] \sphinxstylestrong{\large Credits}
\end{DUlineblock}

\sphinxAtStartPar
This project is created using the open source \sphinxhref{https://jupyterbook.org/}{Jupyter Book project} and the \sphinxhref{https://github.com/executablebooks/cookiecutter-jupyter-book}{executablebooks/cookiecutter\sphinxhyphen{}jupyter\sphinxhyphen{}book template}.

\sphinxAtStartPar
To add: build env \& main software packages (see automation for this…)

\sphinxAtStartPar
\sphinxincludegraphics{{logo}.png}


\chapter{ePSproc base and multijob class intro}
\label{\detokenize{testChpt/ePSproc_class_demo_161020:epsproc-base-and-multijob-class-intro}}\label{\detokenize{testChpt/ePSproc_class_demo_161020::doc}}
\sphinxAtStartPar
16/10/20

\sphinxAtStartPar
As of Oct. 2020, v1.3.0\sphinxhyphen{}dev, basic data classes are now implemented, and are now the easiest/preferred method for using ePSproc (as opposed to calling core functions directly, as \sphinxhref{https://epsproc.readthedocs.io/en/latest/demos/ePSproc\_demo\_Aug2019.html}{illustrated in the functions guide}).

\sphinxAtStartPar
A brief intro and guide to use is given here.

\sphinxAtStartPar
Aims:
\begin{itemize}
\item {} 
\sphinxAtStartPar
Provide unified data architecture for ePSproc, ePSdata and PEMtk.

\item {} 
\sphinxAtStartPar
Wrap plotting and computational functions for ease of use.

\item {} 
\sphinxAtStartPar
Handle multiple datasets inc. comparitive plots.

\end{itemize}


\section{Setup}
\label{\detokenize{testChpt/ePSproc_class_demo_161020:setup}}
\begin{sphinxuseclass}{cell}\begin{sphinxVerbatimInput}

\begin{sphinxuseclass}{cell_input}
\begin{sphinxVerbatim}[commandchars=\\\{\}]
\PYG{c+c1}{\PYGZsh{} For module testing, include path to module here, otherwise use global installation}
\PYG{n}{local} \PYG{o}{=} \PYG{k+kc}{True}

\PYG{k}{if} \PYG{n}{local}\PYG{p}{:}
    \PYG{k+kn}{import} \PYG{n+nn}{sys}
    \PYG{k}{if} \PYG{n}{sys}\PYG{o}{.}\PYG{n}{platform} \PYG{o}{==} \PYG{l+s+s2}{\PYGZdq{}}\PYG{l+s+s2}{win32}\PYG{l+s+s2}{\PYGZdq{}}\PYG{p}{:}
        \PYG{n}{modPath} \PYG{o}{=} \PYG{l+s+sa}{r}\PYG{l+s+s1}{\PYGZsq{}}\PYG{l+s+s1}{D:}\PYG{l+s+s1}{\PYGZbs{}}\PYG{l+s+s1}{code}\PYG{l+s+s1}{\PYGZbs{}}\PYG{l+s+s1}{github}\PYG{l+s+s1}{\PYGZbs{}}\PYG{l+s+s1}{ePSproc}\PYG{l+s+s1}{\PYGZsq{}}  \PYG{c+c1}{\PYGZsh{} Win test machine}
        \PYG{n}{winFlag} \PYG{o}{=} \PYG{k+kc}{True}
    \PYG{k}{else}\PYG{p}{:}
        \PYG{n}{modPath} \PYG{o}{=} \PYG{l+s+sa}{r}\PYG{l+s+s1}{\PYGZsq{}}\PYG{l+s+s1}{/home/femtolab/github/ePSproc/}\PYG{l+s+s1}{\PYGZsq{}}  \PYG{c+c1}{\PYGZsh{} Linux test machine}
        \PYG{n}{winFlag} \PYG{o}{=} \PYG{k+kc}{False}

    \PYG{n}{sys}\PYG{o}{.}\PYG{n}{path}\PYG{o}{.}\PYG{n}{append}\PYG{p}{(}\PYG{n}{modPath}\PYG{p}{)}

\PYG{c+c1}{\PYGZsh{} Base}
\PYG{k+kn}{import} \PYG{n+nn}{epsproc} \PYG{k}{as} \PYG{n+nn}{ep}

\PYG{c+c1}{\PYGZsh{} Class dev code}
\PYG{k+kn}{from} \PYG{n+nn}{epsproc}\PYG{n+nn}{.}\PYG{n+nn}{classes}\PYG{n+nn}{.}\PYG{n+nn}{multiJob} \PYG{k+kn}{import} \PYG{n}{ePSmultiJob}
\PYG{k+kn}{from} \PYG{n+nn}{epsproc}\PYG{n+nn}{.}\PYG{n+nn}{classes}\PYG{n+nn}{.}\PYG{n+nn}{base} \PYG{k+kn}{import} \PYG{n}{ePSbase}
\end{sphinxVerbatim}

\end{sphinxuseclass}\end{sphinxVerbatimInput}
\begin{sphinxVerbatimOutput}

\begin{sphinxuseclass}{cell_output}
\begin{sphinxVerbatim}[commandchars=\\\{\}]
* pyevtk not found, VTK export not available. 
\end{sphinxVerbatim}

\end{sphinxuseclass}\end{sphinxVerbatimOutput}

\end{sphinxuseclass}

\section{ePSbase class}
\label{\detokenize{testChpt/ePSproc_class_demo_161020:epsbase-class}}
\sphinxAtStartPar
The ePSbase class wraps most of the core functionality, and will handle all ePolyScat output files in a single data directory. In general, we’ll assume:
\begin{itemize}
\item {} 
\sphinxAtStartPar
an ePS \sphinxstyleemphasis{job} constitutes a single ionization event/channel (ionizing orbital) for a given molecule, stored in one or more output files.

\item {} 
\sphinxAtStartPar
the data dir contains one or more files, where each file will contain a set of symmetries and energies, with either
\begin{itemize}
\item {} 
\sphinxAtStartPar
one file per ionizing event. In this case, each file will equate to one job, and one entry in the class datastructure.

\item {} 
\sphinxAtStartPar
a single ionizing event, where each file contains a different set of energies for the given event (\sphinxstyleemphasis{energy chunked} fileset). In this case, the files will be stacked, and the dir will equate to one job and one entry in the class datastructure.

\end{itemize}

\end{itemize}

\sphinxAtStartPar
The class datastructure is (currently) a set of dictionaries, with entries per job as above, and various data for each job. In general the data is \sphinxhref{http://xarray.pydata.org/en/stable/why-xarray.html}{stored in Xarrays}.

\sphinxAtStartPar
The \sphinxcode{\sphinxupquote{multiJob}} class extends the base class with reading from multple directories.


\subsection{Load data}
\label{\detokenize{testChpt/ePSproc_class_demo_161020:load-data}}
\sphinxAtStartPar
Firstly, set the data path, instantiate a class object and load the data.

\begin{sphinxuseclass}{cell}\begin{sphinxVerbatimInput}

\begin{sphinxuseclass}{cell_input}
\begin{sphinxVerbatim}[commandchars=\\\{\}]
\PYG{c+c1}{\PYGZsh{} Set for ePSproc test data, available from https://github.com/phockett/ePSproc/tree/master/data}
\PYG{c+c1}{\PYGZsh{} Here this is assumed to be on the epsproc path}
\PYG{k+kn}{import} \PYG{n+nn}{os}
\PYG{n}{dataPath} \PYG{o}{=} \PYG{n}{os}\PYG{o}{.}\PYG{n}{path}\PYG{o}{.}\PYG{n}{join}\PYG{p}{(}\PYG{n}{sys}\PYG{o}{.}\PYG{n}{path}\PYG{p}{[}\PYG{o}{\PYGZhy{}}\PYG{l+m+mi}{1}\PYG{p}{]}\PYG{p}{,} \PYG{l+s+s1}{\PYGZsq{}}\PYG{l+s+s1}{data}\PYG{l+s+s1}{\PYGZsq{}}\PYG{p}{,} \PYG{l+s+s1}{\PYGZsq{}}\PYG{l+s+s1}{photoionization}\PYG{l+s+s1}{\PYGZsq{}}\PYG{p}{,} \PYG{l+s+s1}{\PYGZsq{}}\PYG{l+s+s1}{n2\PYGZus{}multiorb}\PYG{l+s+s1}{\PYGZsq{}}\PYG{p}{)}
\end{sphinxVerbatim}

\end{sphinxuseclass}\end{sphinxVerbatimInput}

\end{sphinxuseclass}
\begin{sphinxuseclass}{cell}\begin{sphinxVerbatimInput}

\begin{sphinxuseclass}{cell_input}
\begin{sphinxVerbatim}[commandchars=\\\{\}]
\PYG{c+c1}{\PYGZsh{} Instantiate class object.}
\PYG{c+c1}{\PYGZsh{} Minimally this needs just the dataPath, if verbose = 1 is set then some useful output will also be printed.}
\PYG{n}{data} \PYG{o}{=} \PYG{n}{ePSbase}\PYG{p}{(}\PYG{n}{dataPath}\PYG{p}{,} \PYG{n}{verbose} \PYG{o}{=} \PYG{l+m+mi}{1}\PYG{p}{)}
\end{sphinxVerbatim}

\end{sphinxuseclass}\end{sphinxVerbatimInput}

\end{sphinxuseclass}
\begin{sphinxuseclass}{cell}\begin{sphinxVerbatimInput}

\begin{sphinxuseclass}{cell_input}
\begin{sphinxVerbatim}[commandchars=\\\{\}]
\PYG{c+c1}{\PYGZsh{} ScanFiles() \PYGZhy{} this will look for data files on the path provided, and read from them.}
\PYG{n}{data}\PYG{o}{.}\PYG{n}{scanFiles}\PYG{p}{(}\PYG{p}{)}
\end{sphinxVerbatim}

\end{sphinxuseclass}\end{sphinxVerbatimInput}
\begin{sphinxVerbatimOutput}

\begin{sphinxuseclass}{cell_output}
\begin{sphinxVerbatim}[commandchars=\\\{\}]
*** Job orb6 details
Key: orb6
Dir D:\PYGZbs{}code\PYGZbs{}github\PYGZbs{}ePSproc\PYGZbs{}data\PYGZbs{}photoionization\PYGZbs{}n2\PYGZus{}multiorb, 1 files.
\PYGZob{}   \PYGZsq{}batch\PYGZsq{}: \PYGZsq{}ePS n2, batch n2\PYGZus{}1pu\PYGZus{}0.1\PYGZhy{}50.1eV, orbital A2\PYGZsq{},
    \PYGZsq{}event\PYGZsq{}: \PYGZsq{} N2 A\PYGZhy{}state (1piu\PYGZhy{}1)\PYGZsq{},
    \PYGZsq{}orbE\PYGZsq{}: \PYGZhy{}17.096913836366,
    \PYGZsq{}orbLabel\PYGZsq{}: \PYGZsq{}1piu\PYGZhy{}1\PYGZsq{}\PYGZcb{}

*** Job orb5 details
Key: orb5
Dir D:\PYGZbs{}code\PYGZbs{}github\PYGZbs{}ePSproc\PYGZbs{}data\PYGZbs{}photoionization\PYGZbs{}n2\PYGZus{}multiorb, 1 files.
\PYGZob{}   \PYGZsq{}batch\PYGZsq{}: \PYGZsq{}ePS n2, batch n2\PYGZus{}3sg\PYGZus{}0.1\PYGZhy{}50.1eV, orbital A2\PYGZsq{},
    \PYGZsq{}event\PYGZsq{}: \PYGZsq{} N2 X\PYGZhy{}state (3sg\PYGZhy{}1)\PYGZsq{},
    \PYGZsq{}orbE\PYGZsq{}: \PYGZhy{}17.341816310545997,
    \PYGZsq{}orbLabel\PYGZsq{}: \PYGZsq{}3sg\PYGZhy{}1\PYGZsq{}\PYGZcb{}
\end{sphinxVerbatim}

\end{sphinxuseclass}\end{sphinxVerbatimOutput}

\end{sphinxuseclass}
\sphinxAtStartPar
In this case, two files are read, and each file is a different ePS job \sphinxhyphen{} here the \(3\sigma_g^{-1}\) and \(1\pi_u^{-1}\) channels in N2. The keys for the job are also used as the job names.


\subsection{Basic info \& plots}
\label{\detokenize{testChpt/ePSproc_class_demo_161020:basic-info-plots}}
\sphinxAtStartPar
A few basic methods to summarise the data…

\begin{sphinxuseclass}{cell}\begin{sphinxVerbatimInput}

\begin{sphinxuseclass}{cell_input}
\begin{sphinxVerbatim}[commandchars=\\\{\}]
\PYG{c+c1}{\PYGZsh{} Summarise jobs, this will also be output by scanFile() if verbose = 1 is set, as illustrated above.}
\PYG{n}{data}\PYG{o}{.}\PYG{n}{jobsSummary}\PYG{p}{(}\PYG{p}{)}
\end{sphinxVerbatim}

\end{sphinxuseclass}\end{sphinxVerbatimInput}
\begin{sphinxVerbatimOutput}

\begin{sphinxuseclass}{cell_output}
\begin{sphinxVerbatim}[commandchars=\\\{\}]
*** Job orb6 details
Key: orb6
Dir D:\PYGZbs{}code\PYGZbs{}github\PYGZbs{}ePSproc\PYGZbs{}data\PYGZbs{}photoionization\PYGZbs{}n2\PYGZus{}multiorb, 1 files.
\PYGZob{}   \PYGZsq{}batch\PYGZsq{}: \PYGZsq{}ePS n2, batch n2\PYGZus{}1pu\PYGZus{}0.1\PYGZhy{}50.1eV, orbital A2\PYGZsq{},
    \PYGZsq{}event\PYGZsq{}: \PYGZsq{} N2 A\PYGZhy{}state (1piu\PYGZhy{}1)\PYGZsq{},
    \PYGZsq{}orbE\PYGZsq{}: \PYGZhy{}17.096913836366,
    \PYGZsq{}orbLabel\PYGZsq{}: \PYGZsq{}1piu\PYGZhy{}1\PYGZsq{}\PYGZcb{}

*** Job orb5 details
Key: orb5
Dir D:\PYGZbs{}code\PYGZbs{}github\PYGZbs{}ePSproc\PYGZbs{}data\PYGZbs{}photoionization\PYGZbs{}n2\PYGZus{}multiorb, 1 files.
\PYGZob{}   \PYGZsq{}batch\PYGZsq{}: \PYGZsq{}ePS n2, batch n2\PYGZus{}3sg\PYGZus{}0.1\PYGZhy{}50.1eV, orbital A2\PYGZsq{},
    \PYGZsq{}event\PYGZsq{}: \PYGZsq{} N2 X\PYGZhy{}state (3sg\PYGZhy{}1)\PYGZsq{},
    \PYGZsq{}orbE\PYGZsq{}: \PYGZhy{}17.341816310545997,
    \PYGZsq{}orbLabel\PYGZsq{}: \PYGZsq{}3sg\PYGZhy{}1\PYGZsq{}\PYGZcb{}
\end{sphinxVerbatim}

\end{sphinxuseclass}\end{sphinxVerbatimOutput}

\end{sphinxuseclass}
\begin{sphinxuseclass}{cell}\begin{sphinxVerbatimInput}

\begin{sphinxuseclass}{cell_input}
\begin{sphinxVerbatim}[commandchars=\\\{\}]
\PYG{c+c1}{\PYGZsh{} Molecular info}
\PYG{c+c1}{\PYGZsh{} Note that this is currently assumed to be the same for all jobs in the data dir.}
\PYG{n}{data}\PYG{o}{.}\PYG{n}{molSummary}\PYG{p}{(}\PYG{p}{)}
\end{sphinxVerbatim}

\end{sphinxuseclass}\end{sphinxVerbatimInput}
\begin{sphinxVerbatimOutput}

\begin{sphinxuseclass}{cell_output}
\begin{sphinxVerbatim}[commandchars=\\\{\}]
*** Molecular structure
\end{sphinxVerbatim}

\noindent\sphinxincludegraphics{{ePSproc_class_demo_161020_12_1}.png}

\begin{sphinxVerbatim}[commandchars=\\\{\}]
*** Molecular orbital list (from ePS output file)
EH = Energy (Hartrees), E = Energy (eV), NOrbGrp, OrbGrp, GrpDegen = degeneracies and corresponding orbital numbering by group in ePS, NormInt = single centre expansion convergence (should be \PYGZti{}1.0).
\end{sphinxVerbatim}

\begin{sphinxVerbatim}[commandchars=\\\{\}]
props Sym       EH  Occ           E  NOrbGrp  OrbGrp  GrpDegen   NormInt
orb                                                                     
1      SG \PYGZhy{}15.6719  2.0 \PYGZhy{}426.454121      1.0     1.0       1.0  0.999532
2      SU \PYGZhy{}15.6676  2.0 \PYGZhy{}426.337112      1.0     2.0       1.0  0.999458
3      SG  \PYGZhy{}1.4948  2.0  \PYGZhy{}40.675580      1.0     3.0       1.0  0.999979
4      SU  \PYGZhy{}0.7687  2.0  \PYGZhy{}20.917392      1.0     4.0       1.0  0.999979
5      SG  \PYGZhy{}0.6373  2.0  \PYGZhy{}17.341816      1.0     5.0       1.0  1.000000
6      PU  \PYGZhy{}0.6283  2.0  \PYGZhy{}17.096914      1.0     6.0       2.0  1.000000
7      PU  \PYGZhy{}0.6283  2.0  \PYGZhy{}17.096914      2.0     6.0       2.0  1.000000
\end{sphinxVerbatim}

\end{sphinxuseclass}\end{sphinxVerbatimOutput}

\end{sphinxuseclass}

\subsection{Plot cross\sphinxhyphen{}sections and betas}
\label{\detokenize{testChpt/ePSproc_class_demo_161020:plot-cross-sections-and-betas}}
\sphinxAtStartPar
These are taken from the \sphinxcode{\sphinxupquote{GetCro}} segments in the ePS output files, and correspond to results for an isotropic ensemble of molecules, i.e. observables in the lab frame (LF) for 1\sphinxhyphen{}photon ionization (see \sphinxhref{https://epsproc.readthedocs.io/en/latest/ePS\_ePSproc\_tutorial/ePS\_tutorial\_080520.html\#Theoretical-background}{the ePS tutorial for more details}).

\begin{sphinxuseclass}{cell}\begin{sphinxVerbatimInput}

\begin{sphinxuseclass}{cell_input}
\begin{sphinxVerbatim}[commandchars=\\\{\}]
\PYG{c+c1}{\PYGZsh{} Minimal method call will plot cross\PYGZhy{}sections for all ePS jobs found in the data directory.}
\PYG{n}{data}\PYG{o}{.}\PYG{n}{plotGetCro}\PYG{p}{(}\PYG{p}{)}
\end{sphinxVerbatim}

\end{sphinxuseclass}\end{sphinxVerbatimInput}
\begin{sphinxVerbatimOutput}

\begin{sphinxuseclass}{cell_output}
\noindent\sphinxincludegraphics{{ePSproc_class_demo_161020_14_0}.png}

\noindent\sphinxincludegraphics{{ePSproc_class_demo_161020_14_1}.png}

\end{sphinxuseclass}\end{sphinxVerbatimOutput}

\end{sphinxuseclass}
\begin{sphinxuseclass}{cell}\begin{sphinxVerbatimInput}

\begin{sphinxuseclass}{cell_input}
\begin{sphinxVerbatim}[commandchars=\\\{\}]
\PYG{c+c1}{\PYGZsh{} Plot beta parameters with the \PYGZsq{}BETA\PYGZsq{} flag}
\PYG{n}{data}\PYG{o}{.}\PYG{n}{plotGetCro}\PYG{p}{(}\PYG{n}{pType} \PYG{o}{=} \PYG{l+s+s1}{\PYGZsq{}}\PYG{l+s+s1}{BETA}\PYG{l+s+s1}{\PYGZsq{}}\PYG{p}{)}
\end{sphinxVerbatim}

\end{sphinxuseclass}\end{sphinxVerbatimInput}
\begin{sphinxVerbatimOutput}

\begin{sphinxuseclass}{cell_output}
\noindent\sphinxincludegraphics{{ePSproc_class_demo_161020_15_0}.png}

\noindent\sphinxincludegraphics{{ePSproc_class_demo_161020_15_1}.png}

\end{sphinxuseclass}\end{sphinxVerbatimOutput}

\end{sphinxuseclass}
\sphinxAtStartPar
TODO: fix labelling here.


\subsection{Compute MFPADs}
\label{\detokenize{testChpt/ePSproc_class_demo_161020:compute-mfpads}}
\sphinxAtStartPar
The class currently wraps just the \sphinxhref{https://epsproc.readthedocs.io/en/latest/demos/ePSproc\_demo\_Aug2019.html\#Calculate-MFPADs}{basic numerical routine for MFPADs}. This defaults to computing MFPADs for all energies and \((z,x,y)\) polarization geometries (where the z\sphinxhyphen{}axis is the molecular symmetry axis, and corresponds to the molecular structure plot shown above).

\begin{sphinxuseclass}{cell}\begin{sphinxVerbatimInput}

\begin{sphinxuseclass}{cell_input}
\begin{sphinxVerbatim}[commandchars=\\\{\}]
\PYG{c+c1}{\PYGZsh{} Compute MFPADs...}
\PYG{n}{data}\PYG{o}{.}\PYG{n}{mfpadNumeric}\PYG{p}{(}\PYG{p}{)}
\end{sphinxVerbatim}

\end{sphinxuseclass}\end{sphinxVerbatimInput}

\end{sphinxuseclass}
\sphinxAtStartPar
To plot it’s advisable to set an enery slice, \sphinxcode{\sphinxupquote{Erange = {[}start, stop, step{]}}}, since MFPADs are currently shown as individual plots in the default case, and there may be a lot of them.

\sphinxAtStartPar
We’ll also just set for a single key here, otherwise all jobs will be plotted.

\begin{sphinxuseclass}{cell}\begin{sphinxVerbatimInput}

\begin{sphinxuseclass}{cell_input}
\begin{sphinxVerbatim}[commandchars=\\\{\}]
\PYG{n}{data}\PYG{o}{.}\PYG{n}{padPlot}\PYG{p}{(}\PYG{n}{keys} \PYG{o}{=} \PYG{l+s+s1}{\PYGZsq{}}\PYG{l+s+s1}{orb5}\PYG{l+s+s1}{\PYGZsq{}}\PYG{p}{,} \PYG{n}{Erange} \PYG{o}{=} \PYG{p}{[}\PYG{l+m+mi}{5}\PYG{p}{,} \PYG{l+m+mi}{10}\PYG{p}{,} \PYG{l+m+mi}{4}\PYG{p}{]}\PYG{p}{)}

\PYG{c+c1}{\PYGZsh{}TODO: fix plot layout!}
\end{sphinxVerbatim}

\end{sphinxuseclass}\end{sphinxVerbatimInput}
\begin{sphinxVerbatimOutput}

\begin{sphinxuseclass}{cell_output}
\begin{sphinxVerbatim}[commandchars=\\\{\}]
Found dims (\PYGZsq{}Labels\PYGZsq{}, \PYGZsq{}Phi\PYGZsq{}, \PYGZsq{}Theta\PYGZsq{}, \PYGZsq{}Eke\PYGZsq{}, \PYGZsq{}Sym\PYGZsq{}), summing to reduce for plot. Pass selDims to avoid.
Sph plots: Pol geom: z
Plotting with mpl
Data dims: (\PYGZsq{}Phi\PYGZsq{}, \PYGZsq{}Theta\PYGZsq{}, \PYGZsq{}Eke\PYGZsq{}), subplots on Eke
Sph plots: Pol geom: x
Plotting with mpl
Data dims: (\PYGZsq{}Phi\PYGZsq{}, \PYGZsq{}Theta\PYGZsq{}, \PYGZsq{}Eke\PYGZsq{}), subplots on Eke
Sph plots: Pol geom: y
Plotting with mpl
Data dims: (\PYGZsq{}Phi\PYGZsq{}, \PYGZsq{}Theta\PYGZsq{}, \PYGZsq{}Eke\PYGZsq{}), subplots on Eke
\end{sphinxVerbatim}

\noindent\sphinxincludegraphics{{ePSproc_class_demo_161020_20_1}.png}

\noindent\sphinxincludegraphics{{ePSproc_class_demo_161020_20_2}.png}

\noindent\sphinxincludegraphics{{ePSproc_class_demo_161020_20_3}.png}

\noindent\sphinxincludegraphics{{ePSproc_class_demo_161020_20_4}.png}

\noindent\sphinxincludegraphics{{ePSproc_class_demo_161020_20_5}.png}

\noindent\sphinxincludegraphics{{ePSproc_class_demo_161020_20_6}.png}

\end{sphinxuseclass}\end{sphinxVerbatimOutput}

\end{sphinxuseclass}
\sphinxAtStartPar
To view multiple results in a more concise fashion, a Cartesian gridded output is also available.

\begin{sphinxuseclass}{cell}\begin{sphinxVerbatimInput}

\begin{sphinxuseclass}{cell_input}
\begin{sphinxVerbatim}[commandchars=\\\{\}]
\PYG{n}{data}\PYG{o}{.}\PYG{n}{padPlot}\PYG{p}{(}\PYG{n}{keys} \PYG{o}{=} \PYG{l+s+s1}{\PYGZsq{}}\PYG{l+s+s1}{orb5}\PYG{l+s+s1}{\PYGZsq{}}\PYG{p}{,} \PYG{n}{Erange} \PYG{o}{=} \PYG{p}{[}\PYG{l+m+mi}{5}\PYG{p}{,} \PYG{l+m+mi}{10}\PYG{p}{,} \PYG{l+m+mi}{4}\PYG{p}{]}\PYG{p}{,} \PYG{n}{pStyle}\PYG{o}{=}\PYG{l+s+s1}{\PYGZsq{}}\PYG{l+s+s1}{grid}\PYG{l+s+s1}{\PYGZsq{}}\PYG{p}{)}
\end{sphinxVerbatim}

\end{sphinxuseclass}\end{sphinxVerbatimInput}
\begin{sphinxVerbatimOutput}

\begin{sphinxuseclass}{cell_output}
\begin{sphinxVerbatim}[commandchars=\\\{\}]
Found dims (\PYGZsq{}Labels\PYGZsq{}, \PYGZsq{}Phi\PYGZsq{}, \PYGZsq{}Theta\PYGZsq{}, \PYGZsq{}Eke\PYGZsq{}, \PYGZsq{}Sym\PYGZsq{}), summing to reduce for plot. Pass selDims to avoid.
Grid plot: 3sg\PYGZhy{}1
\end{sphinxVerbatim}

\noindent\sphinxincludegraphics{{ePSproc_class_demo_161020_22_1}.png}

\end{sphinxuseclass}\end{sphinxVerbatimOutput}

\end{sphinxuseclass}
\begin{sphinxuseclass}{cell}\begin{sphinxVerbatimInput}

\begin{sphinxuseclass}{cell_input}
\begin{sphinxVerbatim}[commandchars=\\\{\}]
\PYG{c+c1}{\PYGZsh{} Various other args can be passed...}

\PYG{c+c1}{\PYGZsh{} Set a plotting backend, currently \PYGZsq{}mpl\PYGZsq{} (Matplotlib \PYGZhy{} default) or \PYGZsq{}pl\PYGZsq{} (Plotly \PYGZhy{} interactive, but may give issues in some environments)}
\PYG{n}{backend} \PYG{o}{=} \PYG{l+s+s1}{\PYGZsq{}}\PYG{l+s+s1}{pl}\PYG{l+s+s1}{\PYGZsq{}}

\PYG{c+c1}{\PYGZsh{} Subselect on dimensions, this is set as a dictionary for Xarray selection (see http://xarray.pydata.org/en/stable/indexing.html\PYGZsh{}indexing\PYGZhy{}with\PYGZhy{}dimension\PYGZhy{}names)}
\PYG{n}{selDims} \PYG{o}{=} \PYG{p}{\PYGZob{}}\PYG{l+s+s1}{\PYGZsq{}}\PYG{l+s+s1}{Labels}\PYG{l+s+s1}{\PYGZsq{}}\PYG{p}{:}\PYG{l+s+s1}{\PYGZsq{}}\PYG{l+s+s1}{z}\PYG{l+s+s1}{\PYGZsq{}}\PYG{p}{\PYGZcb{}}  \PYG{c+c1}{\PYGZsh{} Plot z\PYGZhy{}pol case only.}

\PYG{n}{data}\PYG{o}{.}\PYG{n}{padPlot}\PYG{p}{(}\PYG{n}{Erange} \PYG{o}{=} \PYG{p}{[}\PYG{l+m+mi}{5}\PYG{p}{,} \PYG{l+m+mi}{10}\PYG{p}{,} \PYG{l+m+mi}{4}\PYG{p}{]}\PYG{p}{,} \PYG{n}{selDims}\PYG{o}{=}\PYG{n}{selDims}\PYG{p}{,} \PYG{n}{backend} \PYG{o}{=} \PYG{n}{backend}\PYG{p}{)}
\end{sphinxVerbatim}

\end{sphinxuseclass}\end{sphinxVerbatimInput}
\begin{sphinxVerbatimOutput}

\begin{sphinxuseclass}{cell_output}
\begin{sphinxVerbatim}[commandchars=\\\{\}]
Found dims (\PYGZsq{}Phi\PYGZsq{}, \PYGZsq{}Theta\PYGZsq{}, \PYGZsq{}Eke\PYGZsq{}, \PYGZsq{}Sym\PYGZsq{}), summing to reduce for plot. Pass selDims to avoid.
Sph plots: 1piu\PYGZhy{}1
Plotting with pl
\end{sphinxVerbatim}

\noindent\sphinxincludegraphics{{ePSproc_class_demo_161020_23_2}.png}

\begin{sphinxVerbatim}[commandchars=\\\{\}]
Found dims (\PYGZsq{}Phi\PYGZsq{}, \PYGZsq{}Theta\PYGZsq{}, \PYGZsq{}Eke\PYGZsq{}, \PYGZsq{}Sym\PYGZsq{}), summing to reduce for plot. Pass selDims to avoid.
Sph plots: 3sg\PYGZhy{}1
Plotting with pl
\end{sphinxVerbatim}

\noindent\sphinxincludegraphics{{ePSproc_class_demo_161020_23_4}.png}

\end{sphinxuseclass}\end{sphinxVerbatimOutput}

\end{sphinxuseclass}

\subsection{Compute \protect\(\beta_{LM}\protect\) parameters}
\label{\detokenize{testChpt/ePSproc_class_demo_161020:compute-beta-lm-parameters}}
\sphinxAtStartPar
For computation of \(\beta_{LM}\) parameters the class wraps functions from \sphinxcode{\sphinxupquote{epsproc.geom}}, which implement a tensor method. This is quite fast, although memory heavy, so may not be suitable for very large problems. (See the \sphinxhref{https://epsproc.readthedocs.io/en/latest/methods/geometric\_method\_dev\_260220\_090420\_tidy.html}{method development pages for more info}, more concise notes to follow).
\begin{itemize}
\item {} 
\sphinxAtStartPar
Functions are provided for MF and AF problems (which is the general case, and will equate to the LF case for an unaligned ensemble).

\item {} 
\sphinxAtStartPar
For the MF the class wraps \sphinxcode{\sphinxupquote{ep.geom.mfblmXprod}}, see the \sphinxhref{https://epsproc.readthedocs.io/en/latest/methods/geometric\_method\_dev\_pt2\_170320\_v140420.html}{method development page for more info}, more concise notes to follow.

\item {} 
\sphinxAtStartPar
For the AF the class wraps \sphinxcode{\sphinxupquote{ep.geom.afblmXprod}}, see the \sphinxhref{https://epsproc.readthedocs.io/en/latest/methods/geometric\_method\_dev\_pt3\_AFBLM\_090620\_010920\_dev\_bk100920.html}{method development page for more info}, more concise notes to follow.

\end{itemize}


\subsubsection{Compute MF \protect\(\beta_{LM}\protect\) and PADs}
\label{\detokenize{testChpt/ePSproc_class_demo_161020:compute-mf-beta-lm-and-pads}}
\sphinxAtStartPar
Here’s a quick demo for the default MF cases, which will give parameters corresponding to the \((z,x,y)\) polarization geometries computed by the numerical routine above.

\begin{sphinxuseclass}{cell}\begin{sphinxVerbatimInput}

\begin{sphinxuseclass}{cell_input}
\begin{sphinxVerbatim}[commandchars=\\\{\}]
\PYG{n}{data}\PYG{o}{.}\PYG{n}{MFBLM}\PYG{p}{(}\PYG{p}{)}
\end{sphinxVerbatim}

\end{sphinxuseclass}\end{sphinxVerbatimInput}
\begin{sphinxVerbatimOutput}

\begin{sphinxuseclass}{cell_output}
\begin{sphinxVerbatim}[commandchars=\\\{\}]
Calculating MF\PYGZhy{}BLMs for job key: orb6
Return type BLM.

Calculating MF\PYGZhy{}BLMs for job key: orb5
Return type BLM.
\end{sphinxVerbatim}

\end{sphinxuseclass}\end{sphinxVerbatimOutput}

\end{sphinxuseclass}
\sphinxAtStartPar
Plotting still needs to improve… but ep.lmPlot() is a robust way to plot everything.

\begin{sphinxuseclass}{cell}\begin{sphinxVerbatimInput}

\begin{sphinxuseclass}{cell_input}
\begin{sphinxVerbatim}[commandchars=\\\{\}]
\PYG{c+c1}{\PYGZsh{} data.BLMplot(dataType = \PYGZsq{}MFBLM\PYGZsq{})  \PYGZsh{} HORRIBLE OUTPUT at the moment!!!}
\PYG{n}{data}\PYG{o}{.}\PYG{n}{lmPlot}\PYG{p}{(}\PYG{n}{dataType} \PYG{o}{=} \PYG{l+s+s1}{\PYGZsq{}}\PYG{l+s+s1}{MFBLM}\PYG{l+s+s1}{\PYGZsq{}}\PYG{p}{)}
\end{sphinxVerbatim}

\end{sphinxuseclass}\end{sphinxVerbatimInput}
\begin{sphinxVerbatimOutput}

\begin{sphinxuseclass}{cell_output}
\begin{sphinxVerbatim}[commandchars=\\\{\}]
Plotting data n2\PYGZus{}1pu\PYGZus{}0.1\PYGZhy{}50.1eV\PYGZus{}A2.inp.out, pType=a, thres=0.01, with Seaborn
Plotting data n2\PYGZus{}3sg\PYGZus{}0.1\PYGZhy{}50.1eV\PYGZus{}A2.inp.out, pType=a, thres=0.01, with Seaborn
\end{sphinxVerbatim}

\noindent\sphinxincludegraphics{{ePSproc_class_demo_161020_28_1}.png}

\noindent\sphinxincludegraphics{{ePSproc_class_demo_161020_28_2}.png}

\end{sphinxuseclass}\end{sphinxVerbatimOutput}

\end{sphinxuseclass}
\sphinxAtStartPar
Line\sphinxhyphen{}plots are available with the \sphinxcode{\sphinxupquote{BLMplot}} method, although this currently only supports the Matplotlib backend, and may have problems with dims in some cases (work in progress!).

\begin{sphinxuseclass}{cell}\begin{sphinxVerbatimInput}

\begin{sphinxuseclass}{cell_input}
\begin{sphinxVerbatim}[commandchars=\\\{\}]
\PYG{n}{data}\PYG{o}{.}\PYG{n}{BLMplot}\PYG{p}{(}\PYG{n}{dataType}\PYG{o}{=}\PYG{l+s+s1}{\PYGZsq{}}\PYG{l+s+s1}{MFBLM}\PYG{l+s+s1}{\PYGZsq{}}\PYG{p}{,} \PYG{n}{thres} \PYG{o}{=} \PYG{l+m+mf}{1e\PYGZhy{}2}\PYG{p}{)}  \PYG{c+c1}{\PYGZsh{} Passing a threshold value here will remove any spurious BLM parameters.}
\end{sphinxVerbatim}

\end{sphinxuseclass}\end{sphinxVerbatimInput}
\begin{sphinxVerbatimOutput}

\begin{sphinxuseclass}{cell_output}
\begin{sphinxVerbatim}[commandchars=\\\{\}]
Dataset: orb6, 1piu\PYGZhy{}1, MFBLM
Dataset: orb5, 3sg\PYGZhy{}1, MFBLM
\end{sphinxVerbatim}

\noindent\sphinxincludegraphics{{ePSproc_class_demo_161020_30_1}.png}

\noindent\sphinxincludegraphics{{ePSproc_class_demo_161020_30_2}.png}

\end{sphinxuseclass}\end{sphinxVerbatimOutput}

\end{sphinxuseclass}
\sphinxAtStartPar
Polar plots are available for these distributions using the \sphinxcode{\sphinxupquote{padPlot()}} method if the \sphinxcode{\sphinxupquote{dataType}} is passed.

\begin{sphinxuseclass}{cell}\begin{sphinxVerbatimInput}

\begin{sphinxuseclass}{cell_input}
\begin{sphinxVerbatim}[commandchars=\\\{\}]
\PYG{n}{data}\PYG{o}{.}\PYG{n}{padPlot}\PYG{p}{(}\PYG{n}{keys} \PYG{o}{=} \PYG{l+s+s1}{\PYGZsq{}}\PYG{l+s+s1}{orb5}\PYG{l+s+s1}{\PYGZsq{}}\PYG{p}{,} \PYG{n}{Erange} \PYG{o}{=} \PYG{p}{[}\PYG{l+m+mi}{5}\PYG{p}{,} \PYG{l+m+mi}{10}\PYG{p}{,} \PYG{l+m+mi}{4}\PYG{p}{]}\PYG{p}{,} \PYG{n}{dataType}\PYG{o}{=}\PYG{l+s+s1}{\PYGZsq{}}\PYG{l+s+s1}{MFBLM}\PYG{l+s+s1}{\PYGZsq{}}\PYG{p}{)}
\end{sphinxVerbatim}

\end{sphinxuseclass}\end{sphinxVerbatimInput}
\begin{sphinxVerbatimOutput}

\begin{sphinxuseclass}{cell_output}
\begin{sphinxVerbatim}[commandchars=\\\{\}]
Using default sph betas.
Sph plots: Pol geom: z
Plotting with mpl
Data dims: (\PYGZsq{}Eke\PYGZsq{}, \PYGZsq{}Phi\PYGZsq{}, \PYGZsq{}Theta\PYGZsq{}), subplots on Eke
Sph plots: Pol geom: x
Plotting with mpl
Data dims: (\PYGZsq{}Eke\PYGZsq{}, \PYGZsq{}Phi\PYGZsq{}, \PYGZsq{}Theta\PYGZsq{}), subplots on Eke
Sph plots: Pol geom: y
Plotting with mpl
Data dims: (\PYGZsq{}Eke\PYGZsq{}, \PYGZsq{}Phi\PYGZsq{}, \PYGZsq{}Theta\PYGZsq{}), subplots on Eke
\end{sphinxVerbatim}

\noindent\sphinxincludegraphics{{ePSproc_class_demo_161020_32_1}.png}

\noindent\sphinxincludegraphics{{ePSproc_class_demo_161020_32_2}.png}

\noindent\sphinxincludegraphics{{ePSproc_class_demo_161020_32_3}.png}

\noindent\sphinxincludegraphics{{ePSproc_class_demo_161020_32_4}.png}

\noindent\sphinxincludegraphics{{ePSproc_class_demo_161020_32_5}.png}

\noindent\sphinxincludegraphics{{ePSproc_class_demo_161020_32_6}.png}

\end{sphinxuseclass}\end{sphinxVerbatimOutput}

\end{sphinxuseclass}

\subsubsection{Compute LF/AF \protect\(\beta_{LM}\protect\) and PADs}
\label{\detokenize{testChpt/ePSproc_class_demo_161020:compute-lf-af-beta-lm-and-pads}}
\sphinxAtStartPar
Here’s a quick demo for the default AF case (isotropic distribution, hence == LF case).

\begin{sphinxuseclass}{cell}\begin{sphinxVerbatimInput}

\begin{sphinxuseclass}{cell_input}
\begin{sphinxVerbatim}[commandchars=\\\{\}]
\PYG{n}{data}\PYG{o}{.}\PYG{n}{AFBLM}\PYG{p}{(}\PYG{p}{)}
\end{sphinxVerbatim}

\end{sphinxuseclass}\end{sphinxVerbatimInput}
\begin{sphinxVerbatimOutput}

\begin{sphinxuseclass}{cell_output}
\begin{sphinxVerbatim}[commandchars=\\\{\}]
Calculating AF\PYGZhy{}BLMs for job key: orb6

Calculating AF\PYGZhy{}BLMs for job key: orb5
\end{sphinxVerbatim}

\end{sphinxuseclass}\end{sphinxVerbatimOutput}

\end{sphinxuseclass}
\sphinxAtStartPar
Plotting still needs to improve… but ep.lmPlot() is a robust way to plot everything.

\begin{sphinxuseclass}{cell}\begin{sphinxVerbatimInput}

\begin{sphinxuseclass}{cell_input}
\begin{sphinxVerbatim}[commandchars=\\\{\}]
\PYG{c+c1}{\PYGZsh{} data.BLMplot(dataType = \PYGZsq{}MFBLM\PYGZsq{})  \PYGZsh{} HORRIBLE OUTPUT at the moment!!!}
\PYG{n}{data}\PYG{o}{.}\PYG{n}{lmPlot}\PYG{p}{(}\PYG{n}{dataType} \PYG{o}{=} \PYG{l+s+s1}{\PYGZsq{}}\PYG{l+s+s1}{AFBLM}\PYG{l+s+s1}{\PYGZsq{}}\PYG{p}{)}
\end{sphinxVerbatim}

\end{sphinxuseclass}\end{sphinxVerbatimInput}
\begin{sphinxVerbatimOutput}

\begin{sphinxuseclass}{cell_output}
\begin{sphinxVerbatim}[commandchars=\\\{\}]
Plotting data n2\PYGZus{}1pu\PYGZus{}0.1\PYGZhy{}50.1eV\PYGZus{}A2.inp.out, pType=a, thres=0.01, with Seaborn
Plotting data n2\PYGZus{}3sg\PYGZus{}0.1\PYGZhy{}50.1eV\PYGZus{}A2.inp.out, pType=a, thres=0.01, with Seaborn
\end{sphinxVerbatim}

\noindent\sphinxincludegraphics{{ePSproc_class_demo_161020_36_1}.png}

\noindent\sphinxincludegraphics{{ePSproc_class_demo_161020_36_2}.png}

\end{sphinxuseclass}\end{sphinxVerbatimOutput}

\end{sphinxuseclass}
\sphinxAtStartPar
Line\sphinxhyphen{}plots are available with the \sphinxcode{\sphinxupquote{BLMplot}} method, although this currently only supports the Matplotlib backend, and may have problems with dims in some cases (work in progress!).

\begin{sphinxuseclass}{cell}\begin{sphinxVerbatimInput}

\begin{sphinxuseclass}{cell_input}
\begin{sphinxVerbatim}[commandchars=\\\{\}]
\PYG{n}{data}\PYG{o}{.}\PYG{n}{BLMplot}\PYG{p}{(}\PYG{n}{dataType}\PYG{o}{=}\PYG{l+s+s1}{\PYGZsq{}}\PYG{l+s+s1}{AFBLM}\PYG{l+s+s1}{\PYGZsq{}}\PYG{p}{,} \PYG{n}{thres} \PYG{o}{=} \PYG{l+m+mf}{1e\PYGZhy{}2}\PYG{p}{)}  \PYG{c+c1}{\PYGZsh{} Passing a threshold value here will remove any spurious BLM parameters.}
\end{sphinxVerbatim}

\end{sphinxuseclass}\end{sphinxVerbatimInput}
\begin{sphinxVerbatimOutput}

\begin{sphinxuseclass}{cell_output}
\begin{sphinxVerbatim}[commandchars=\\\{\}]
Dataset: orb6, 1piu\PYGZhy{}1, XS
Dataset: orb6, 1piu\PYGZhy{}1, AFBLM
Dataset: orb5, 3sg\PYGZhy{}1, XS
Dataset: orb5, 3sg\PYGZhy{}1, AFBLM
\end{sphinxVerbatim}

\noindent\sphinxincludegraphics{{ePSproc_class_demo_161020_38_1}.png}

\noindent\sphinxincludegraphics{{ePSproc_class_demo_161020_38_2}.png}

\noindent\sphinxincludegraphics{{ePSproc_class_demo_161020_38_3}.png}

\noindent\sphinxincludegraphics{{ePSproc_class_demo_161020_38_4}.png}

\end{sphinxuseclass}\end{sphinxVerbatimOutput}

\end{sphinxuseclass}
\sphinxAtStartPar
Polar plots are available for these distributions using the \sphinxcode{\sphinxupquote{padPlot()}} method.

\begin{sphinxuseclass}{cell}\begin{sphinxVerbatimInput}

\begin{sphinxuseclass}{cell_input}
\begin{sphinxVerbatim}[commandchars=\\\{\}]
\PYG{n}{data}\PYG{o}{.}\PYG{n}{padPlot}\PYG{p}{(}\PYG{n}{keys} \PYG{o}{=} \PYG{l+s+s1}{\PYGZsq{}}\PYG{l+s+s1}{orb5}\PYG{l+s+s1}{\PYGZsq{}}\PYG{p}{,} \PYG{n}{Erange} \PYG{o}{=} \PYG{p}{[}\PYG{l+m+mi}{5}\PYG{p}{,} \PYG{l+m+mi}{10}\PYG{p}{,} \PYG{l+m+mi}{4}\PYG{p}{]}\PYG{p}{,} \PYG{n}{dataType}\PYG{o}{=}\PYG{l+s+s1}{\PYGZsq{}}\PYG{l+s+s1}{AFBLM}\PYG{l+s+s1}{\PYGZsq{}}\PYG{p}{)}

\PYG{c+c1}{\PYGZsh{} NOTE \PYGZhy{} seem to have an inconsistency with (x,y) pol geometries here \PYGZhy{} should check source code \PYGZam{} fix. Likely due to mix\PYGZhy{}up in frame defns., i.e. probably mixing LF and MF pol geom defn. \PYGZhy{} TBC.}
\end{sphinxVerbatim}

\end{sphinxuseclass}\end{sphinxVerbatimInput}
\begin{sphinxVerbatimOutput}

\begin{sphinxuseclass}{cell_output}
\begin{sphinxVerbatim}[commandchars=\\\{\}]
Found dims (\PYGZsq{}Labels\PYGZsq{}, \PYGZsq{}t\PYGZsq{}, \PYGZsq{}Eke\PYGZsq{}, \PYGZsq{}BLM\PYGZsq{}), summing to reduce for plot. Pass selDims to avoid.
Using default sph betas.
Sph plots: Pol geom: z
Plotting with mpl
Data dims: (\PYGZsq{}t\PYGZsq{}, \PYGZsq{}Eke\PYGZsq{}, \PYGZsq{}Phi\PYGZsq{}, \PYGZsq{}Theta\PYGZsq{}), subplots on Eke
Sph plots: Pol geom: x
Plotting with mpl
Data dims: (\PYGZsq{}t\PYGZsq{}, \PYGZsq{}Eke\PYGZsq{}, \PYGZsq{}Phi\PYGZsq{}, \PYGZsq{}Theta\PYGZsq{}), subplots on Eke
Sph plots: Pol geom: y
Plotting with mpl
Data dims: (\PYGZsq{}t\PYGZsq{}, \PYGZsq{}Eke\PYGZsq{}, \PYGZsq{}Phi\PYGZsq{}, \PYGZsq{}Theta\PYGZsq{}), subplots on Eke
\end{sphinxVerbatim}

\noindent\sphinxincludegraphics{{ePSproc_class_demo_161020_40_1}.png}

\noindent\sphinxincludegraphics{{ePSproc_class_demo_161020_40_2}.png}

\noindent\sphinxincludegraphics{{ePSproc_class_demo_161020_40_3}.png}

\noindent\sphinxincludegraphics{{ePSproc_class_demo_161020_40_4}.png}

\noindent\sphinxincludegraphics{{ePSproc_class_demo_161020_40_5}.png}

\noindent\sphinxincludegraphics{{ePSproc_class_demo_161020_40_6}.png}

\end{sphinxuseclass}\end{sphinxVerbatimOutput}

\end{sphinxuseclass}

\section{Additions}
\label{\detokenize{testChpt/ePSproc_class_demo_161020:additions}}

\subsection{Plot styles for line\sphinxhyphen{}plots}
\label{\detokenize{testChpt/ePSproc_class_demo_161020:plot-styles-for-line-plots}}
\sphinxAtStartPar
To set to Seaborn plotting style, use \sphinxcode{\sphinxupquote{ep.hvPlotters.setPlotters()}} (note this will be set for all plots after loading, unless overriden). Seaborn must be installed for this to function.

\sphinxAtStartPar
For more on Seaborn styles, see \sphinxhref{https://seaborn.pydata.org/tutorial/aesthetics.html}{the Seaborn docs}.

\begin{sphinxuseclass}{cell}\begin{sphinxVerbatimInput}

\begin{sphinxuseclass}{cell_input}
\begin{sphinxVerbatim}[commandchars=\\\{\}]
\PYG{k+kn}{from} \PYG{n+nn}{epsproc}\PYG{n+nn}{.}\PYG{n+nn}{plot} \PYG{k+kn}{import} \PYG{n}{hvPlotters}
\PYG{n}{hvPlotters}\PYG{o}{.}\PYG{n}{setPlotters}\PYG{p}{(}\PYG{p}{)}

\PYG{n}{data}\PYG{o}{.}\PYG{n}{plotGetCro}\PYG{p}{(}\PYG{p}{)}
\end{sphinxVerbatim}

\end{sphinxuseclass}\end{sphinxVerbatimInput}
\begin{sphinxVerbatimOutput}

\begin{sphinxuseclass}{cell_output}
\noindent\sphinxincludegraphics{{ePSproc_class_demo_161020_43_3}.png}

\noindent\sphinxincludegraphics{{ePSproc_class_demo_161020_43_4}.png}

\end{sphinxuseclass}\end{sphinxVerbatimOutput}

\end{sphinxuseclass}

\subsection{Matrix element plotting}
\label{\detokenize{testChpt/ePSproc_class_demo_161020:matrix-element-plotting}}
\sphinxAtStartPar
For a full view of the computational results, use the \sphinxcode{\sphinxupquote{lmPlot()}} method with the default, which correspond to \sphinxcode{\sphinxupquote{dataType=matE}}.

\begin{sphinxuseclass}{cell}\begin{sphinxVerbatimInput}

\begin{sphinxuseclass}{cell_input}
\begin{sphinxVerbatim}[commandchars=\\\{\}]
\PYG{n}{data}\PYG{o}{.}\PYG{n}{lmPlot}\PYG{p}{(}\PYG{p}{)}
\end{sphinxVerbatim}

\end{sphinxuseclass}\end{sphinxVerbatimInput}
\begin{sphinxVerbatimOutput}

\begin{sphinxuseclass}{cell_output}
\begin{sphinxVerbatim}[commandchars=\\\{\}]
Plotting data n2\PYGZus{}1pu\PYGZus{}0.1\PYGZhy{}50.1eV\PYGZus{}A2.inp.out, pType=a, thres=0.01, with Seaborn
Plotting data n2\PYGZus{}3sg\PYGZus{}0.1\PYGZhy{}50.1eV\PYGZus{}A2.inp.out, pType=a, thres=0.01, with Seaborn
\end{sphinxVerbatim}

\noindent\sphinxincludegraphics{{ePSproc_class_demo_161020_45_1}.png}

\noindent\sphinxincludegraphics{{ePSproc_class_demo_161020_45_2}.png}

\end{sphinxuseclass}\end{sphinxVerbatimOutput}

\end{sphinxuseclass}
\sphinxAtStartPar
The default here plots abs values, but the same routine can be set for phase plotting.

\begin{sphinxuseclass}{cell}\begin{sphinxVerbatimInput}

\begin{sphinxuseclass}{cell_input}
\begin{sphinxVerbatim}[commandchars=\\\{\}]
\PYG{n}{data}\PYG{o}{.}\PYG{n}{lmPlot}\PYG{p}{(}\PYG{n}{pType}\PYG{o}{=}\PYG{l+s+s1}{\PYGZsq{}}\PYG{l+s+s1}{phase}\PYG{l+s+s1}{\PYGZsq{}}\PYG{p}{)}
\end{sphinxVerbatim}

\end{sphinxuseclass}\end{sphinxVerbatimInput}
\begin{sphinxVerbatimOutput}

\begin{sphinxuseclass}{cell_output}
\begin{sphinxVerbatim}[commandchars=\\\{\}]
Plotting data n2\PYGZus{}1pu\PYGZus{}0.1\PYGZhy{}50.1eV\PYGZus{}A2.inp.out, pType=phase, thres=0.01, with Seaborn
Plotting data n2\PYGZus{}3sg\PYGZus{}0.1\PYGZhy{}50.1eV\PYGZus{}A2.inp.out, pType=phase, thres=0.01, with Seaborn
\end{sphinxVerbatim}

\noindent\sphinxincludegraphics{{ePSproc_class_demo_161020_47_1}.png}

\noindent\sphinxincludegraphics{{ePSproc_class_demo_161020_47_2}.png}

\end{sphinxuseclass}\end{sphinxVerbatimOutput}

\end{sphinxuseclass}

\section{Versions}
\label{\detokenize{testChpt/ePSproc_class_demo_161020:versions}}
\begin{sphinxuseclass}{cell}\begin{sphinxVerbatimInput}

\begin{sphinxuseclass}{cell_input}
\begin{sphinxVerbatim}[commandchars=\\\{\}]
\PYG{k+kn}{import} \PYG{n+nn}{scooby}
\PYG{n}{scooby}\PYG{o}{.}\PYG{n}{Report}\PYG{p}{(}\PYG{n}{additional}\PYG{o}{=}\PYG{p}{[}\PYG{l+s+s1}{\PYGZsq{}}\PYG{l+s+s1}{epsproc}\PYG{l+s+s1}{\PYGZsq{}}\PYG{p}{,} \PYG{l+s+s1}{\PYGZsq{}}\PYG{l+s+s1}{xarray}\PYG{l+s+s1}{\PYGZsq{}}\PYG{p}{,} \PYG{l+s+s1}{\PYGZsq{}}\PYG{l+s+s1}{jupyter}\PYG{l+s+s1}{\PYGZsq{}}\PYG{p}{]}\PYG{p}{)}
\end{sphinxVerbatim}

\end{sphinxuseclass}\end{sphinxVerbatimInput}
\begin{sphinxVerbatimOutput}

\begin{sphinxuseclass}{cell_output}
\begin{sphinxVerbatim}[commandchars=\\\{\}]
\PYGZhy{}\PYGZhy{}\PYGZhy{}\PYGZhy{}\PYGZhy{}\PYGZhy{}\PYGZhy{}\PYGZhy{}\PYGZhy{}\PYGZhy{}\PYGZhy{}\PYGZhy{}\PYGZhy{}\PYGZhy{}\PYGZhy{}\PYGZhy{}\PYGZhy{}\PYGZhy{}\PYGZhy{}\PYGZhy{}\PYGZhy{}\PYGZhy{}\PYGZhy{}\PYGZhy{}\PYGZhy{}\PYGZhy{}\PYGZhy{}\PYGZhy{}\PYGZhy{}\PYGZhy{}\PYGZhy{}\PYGZhy{}\PYGZhy{}\PYGZhy{}\PYGZhy{}\PYGZhy{}\PYGZhy{}\PYGZhy{}\PYGZhy{}\PYGZhy{}\PYGZhy{}\PYGZhy{}\PYGZhy{}\PYGZhy{}\PYGZhy{}\PYGZhy{}\PYGZhy{}\PYGZhy{}\PYGZhy{}\PYGZhy{}\PYGZhy{}\PYGZhy{}\PYGZhy{}\PYGZhy{}\PYGZhy{}\PYGZhy{}\PYGZhy{}\PYGZhy{}\PYGZhy{}\PYGZhy{}\PYGZhy{}\PYGZhy{}\PYGZhy{}\PYGZhy{}\PYGZhy{}\PYGZhy{}\PYGZhy{}\PYGZhy{}\PYGZhy{}\PYGZhy{}\PYGZhy{}\PYGZhy{}\PYGZhy{}\PYGZhy{}\PYGZhy{}\PYGZhy{}\PYGZhy{}\PYGZhy{}\PYGZhy{}\PYGZhy{}
  Date: Tue Oct 20 15:57:33 2020 Eastern Daylight Time

                OS : Windows
            CPU(s) : 32
           Machine : AMD64
      Architecture : 64bit
               RAM : 63.9 GB
       Environment : Jupyter

  Python 3.7.3 (default, Apr 24 2019, 15:29:51) [MSC v.1915 64 bit (AMD64)]

           epsproc : 1.3.0\PYGZhy{}dev
            xarray : 0.15.0
           jupyter : Version unknown
             numpy : 1.19.2
             scipy : 1.3.0
           IPython : 7.12.0
        matplotlib : 3.3.1
            scooby : 0.5.6

  Intel(R) Math Kernel Library Version 2020.0.0 Product Build 20191125 for
  Intel(R) 64 architecture applications
\PYGZhy{}\PYGZhy{}\PYGZhy{}\PYGZhy{}\PYGZhy{}\PYGZhy{}\PYGZhy{}\PYGZhy{}\PYGZhy{}\PYGZhy{}\PYGZhy{}\PYGZhy{}\PYGZhy{}\PYGZhy{}\PYGZhy{}\PYGZhy{}\PYGZhy{}\PYGZhy{}\PYGZhy{}\PYGZhy{}\PYGZhy{}\PYGZhy{}\PYGZhy{}\PYGZhy{}\PYGZhy{}\PYGZhy{}\PYGZhy{}\PYGZhy{}\PYGZhy{}\PYGZhy{}\PYGZhy{}\PYGZhy{}\PYGZhy{}\PYGZhy{}\PYGZhy{}\PYGZhy{}\PYGZhy{}\PYGZhy{}\PYGZhy{}\PYGZhy{}\PYGZhy{}\PYGZhy{}\PYGZhy{}\PYGZhy{}\PYGZhy{}\PYGZhy{}\PYGZhy{}\PYGZhy{}\PYGZhy{}\PYGZhy{}\PYGZhy{}\PYGZhy{}\PYGZhy{}\PYGZhy{}\PYGZhy{}\PYGZhy{}\PYGZhy{}\PYGZhy{}\PYGZhy{}\PYGZhy{}\PYGZhy{}\PYGZhy{}\PYGZhy{}\PYGZhy{}\PYGZhy{}\PYGZhy{}\PYGZhy{}\PYGZhy{}\PYGZhy{}\PYGZhy{}\PYGZhy{}\PYGZhy{}\PYGZhy{}\PYGZhy{}\PYGZhy{}\PYGZhy{}\PYGZhy{}\PYGZhy{}\PYGZhy{}\PYGZhy{}
\end{sphinxVerbatim}

\end{sphinxuseclass}\end{sphinxVerbatimOutput}

\end{sphinxuseclass}

\section{ePSproc Matlab demo}
\label{\detokenize{testChpt/ePSproc_Matlab_demo_notebook_090821:epsproc-matlab-demo}}\label{\detokenize{testChpt/ePSproc_Matlab_demo_notebook_090821::doc}}
\sphinxAtStartPar
As noted in the \DUrole{xref,myst}{main readme/intro doc}, ePSproc originally started as a set of Matlab routines, before migration to Python in 2019. Although no longer maintained/updated since \textasciitilde{}2018, the Matlab routines are still available as part of the ePSproc distribution. This notebook demos the routines, following the demo script \sphinxcode{\sphinxupquote{ePSproc\_NO2\_MFPADs\_demo.m}} (to run this code in a Jupyter environment, a Matlab kernel is required, e.g. \sphinxhref{https://github.com/Calysto/matlab\_kernel}{Calysto’s Matlab kernel}).

\sphinxAtStartPar
Functionality:
\begin{itemize}
\item {} 
\sphinxAtStartPar
Read raw photoionization matrix elements from ePS output files with “dumpIdy” segments

\item {} 
\sphinxAtStartPar
Calculate MF\sphinxhyphen{}PADs from the matrix elements (ePSproc\_MFPAD.m, see also ePSproc\_NO2\_MFPADs\_demo.m)

\item {} 
\sphinxAtStartPar
Plot MF\sphinxhyphen{}PADs

\item {} 
\sphinxAtStartPar
Plot X\sphinxhyphen{}sects

\item {} 
\sphinxAtStartPar
(Beta testing): Calculate MF\sphinxhyphen{}BLMs from matrix elements, see ePSproc\_MFBLM.m

\item {} 
\sphinxAtStartPar
(Under development): Calculate AF\sphinxhyphen{}BLMs from matrix elements.

\end{itemize}

\sphinxAtStartPar
Source:
\begin{itemize}
\item {} 
\sphinxAtStartPar
/matlab: stable matlab code (as per \sphinxcode{\sphinxupquote{release v1.0.1 <https://github.com/phockett/ePSproc/releases>}}\_\_).
\begin{itemize}
\item {} 
\sphinxAtStartPar
a set of functions for processing (ePSproc*.m files)

\item {} 
\sphinxAtStartPar
a script showing demo calculations, \sphinxcode{\sphinxupquote{ePSproc\_NO2\_MFPADs\_demo.m}}

\end{itemize}

\item {} 
\sphinxAtStartPar
/docs/additional contains:
\begin{itemize}
\item {} 
\sphinxAtStartPar
the benchmark results from these calculations, \sphinxcode{\sphinxupquote{ePSproc\_NO2\_testing\_summary\_250915.pdf}}

\item {} 
\sphinxAtStartPar
additional notes on ePS photoionization matrix elements, \sphinxcode{\sphinxupquote{ePSproc\_scattering\_theory\_ePS\_notes\_011015.pdf}}.

\end{itemize}

\end{itemize}

\sphinxAtStartPar
See \sphinxcode{\sphinxupquote{ePSproc: Post\sphinxhyphen{}processing suite for ePolyScat electron\sphinxhyphen{}molecule scattering calculations <https://www.authorea.com/users/71114/articles/122402/\_show\_article>}}\_ for more details.


\subsection{Setup}
\label{\detokenize{testChpt/ePSproc_Matlab_demo_notebook_090821:setup}}
\begin{sphinxuseclass}{cell}\begin{sphinxVerbatimInput}

\begin{sphinxuseclass}{cell_input}
\begin{sphinxVerbatim}[commandchars=\\\{\}]
\PYG{n+nb}{cd}\PYG{p}{(}\PYG{l+s}{\PYGZsq{}}\PYG{l+s}{\PYGZti{}/github/ePSproc\PYGZsq{}}\PYG{p}{)}\PYG{+w}{  }\PYG{c}{\PYGZpc{} Change to ePSproc root dir.}
\end{sphinxVerbatim}

\end{sphinxuseclass}\end{sphinxVerbatimInput}
\begin{sphinxVerbatimOutput}

\begin{sphinxuseclass}{cell_output}
\begin{sphinxVerbatim}[commandchars=\\\{\}]
ans =

    \PYGZsq{}9.4.0.813654 (R2018a)\PYGZsq{}
\end{sphinxVerbatim}

\end{sphinxuseclass}\end{sphinxVerbatimOutput}

\end{sphinxuseclass}
\begin{sphinxuseclass}{cell}\begin{sphinxVerbatimInput}

\begin{sphinxuseclass}{cell_input}
\begin{sphinxVerbatim}[commandchars=\\\{\}]
\PYG{c}{\PYGZpc{}\PYGZpc{} *** SETTINGS}
\PYG{c}{\PYGZpc{}  Set up basic environment}

\PYG{c}{\PYGZpc{} Name \PYGZam{} path to ePS output file. In this version set full file name here, and working directory below.}
\PYG{c}{\PYGZpc{} fileName=\PYGZsq{}no2\PYGZus{}demo\PYGZus{}ePS.out\PYGZsq{}  \PYGZpc{} OK for MFPAD testing, but only single E point}
\PYG{n+nb}{fileName}\PYG{p}{=}\PYG{l+s}{\PYGZsq{}}\PYG{l+s}{n2\PYGZus{}3sg\PYGZus{}0.1\PYGZhy{}50.1eV\PYGZus{}A2.inp.out\PYGZsq{}}

\PYG{c}{\PYGZpc{} Set paths for Linux or Win boxes (optional!)}
\PYG{k}{if}\PYG{+w}{ }\PYG{n+nb}{isunix}
\PYG{+w}{    }\PYG{n}{dirSlash}\PYG{p}{=}\PYG{l+s}{\PYGZsq{}}\PYG{l+s}{/\PYGZsq{}}\PYG{p}{;}
\PYG{k}{else}
\PYG{+w}{    }\PYG{n}{dirSlash}\PYG{p}{=}\PYG{l+s}{\PYGZsq{}}\PYG{l+s}{\PYGZbs{}\PYGZsq{}}\PYG{p}{;}
\PYG{k}{end}

\PYG{n}{filePath}\PYG{p}{=}\PYG{p}{[}\PYG{n+nb}{pwd}\PYG{+w}{ }\PYG{n}{dirSlash}\PYG{+w}{ }\PYG{l+s}{\PYGZsq{}}\PYG{l+s}{data\PYGZsq{}}\PYG{+w}{ }\PYG{n}{dirSlash}\PYG{+w}{ }\PYG{l+s}{\PYGZsq{}}\PYG{l+s}{photoionization\PYGZsq{}}\PYG{p}{]}\PYG{p}{;}\PYG{+w}{                       }\PYG{c}{\PYGZpc{} Root to working directory, here set as current dir/data/photoionization}
\PYG{n}{fileBase}\PYG{p}{=}\PYG{p}{[}\PYG{n}{filePath}\PYG{+w}{ }\PYG{n}{dirSlash}\PYG{+w}{ }\PYG{n+nb}{fileName}\PYG{p}{]}\PYG{p}{;}\PYG{+w}{   }\PYG{c}{\PYGZpc{} Full path to ePS results file, here set as current working direcory}

\PYG{n}{scriptPath}\PYG{p}{=}\PYG{p}{[}\PYG{n+nb}{pwd}\PYG{+w}{ }\PYG{n}{dirSlash}\PYG{+w}{ }\PYG{l+s}{\PYGZsq{}}\PYG{l+s}{matlab\PYGZsq{}}\PYG{+w}{ }\PYG{n}{dirSlash}\PYG{p}{]}\PYG{p}{;}\PYG{+w}{  }\PYG{c}{\PYGZpc{} Add path to ePSproc scrips to Matlab path list, here set as current dir/matlab}
\PYG{n+nb}{path}\PYG{p}{(}\PYG{n+nb}{path}\PYG{p}{,}\PYG{p}{[}\PYG{n}{scriptPath}\PYG{p}{]}\PYG{p}{)}\PYG{p}{;}\PYG{+w}{   }
\end{sphinxVerbatim}

\end{sphinxuseclass}\end{sphinxVerbatimInput}
\begin{sphinxVerbatimOutput}

\begin{sphinxuseclass}{cell_output}
\begin{sphinxVerbatim}[commandchars=\\\{\}]
fileName =

    \PYGZsq{}n2\PYGZus{}3sg\PYGZus{}0.1\PYGZhy{}50.1eV\PYGZus{}A2.inp.out\PYGZsq{}
\end{sphinxVerbatim}

\end{sphinxuseclass}\end{sphinxVerbatimOutput}

\end{sphinxuseclass}
\begin{sphinxuseclass}{cell}\begin{sphinxVerbatimInput}

\begin{sphinxuseclass}{cell_input}
\begin{sphinxVerbatim}[commandchars=\\\{\}]
\PYG{c}{\PYGZpc{}\PYGZpc{} *** Read data}
\PYG{c}{\PYGZpc{}  Variables:}
\PYG{c}{\PYGZpc{}       rlAll contains matrix elements (from DumpIdy segments)}
\PYG{c}{\PYGZpc{}       params contains various calculation parameters}
\PYG{c}{\PYGZpc{}       getCro contains cross\PYGZhy{}section (from GetCro segments), if present}

\PYG{p}{[}\PYG{n}{rlAll}\PYG{p}{,}\PYG{+w}{ }\PYG{n}{params}\PYG{p}{,}\PYG{+w}{ }\PYG{n}{getCro}\PYG{p}{]}\PYG{p}{=}\PYG{n}{ePSproc\PYGZus{}read}\PYG{p}{(}\PYG{n}{fileBase}\PYG{p}{)}\PYG{p}{;}

\PYG{n}{params}\PYG{p}{.}\PYG{n}{fileBase}\PYG{p}{=}\PYG{n}{fileBase}\PYG{p}{;}
\PYG{n}{params}\PYG{p}{.}\PYG{n}{fileName}\PYG{p}{=}\PYG{n+nb}{fileName}\PYG{p}{;}
\end{sphinxVerbatim}

\end{sphinxuseclass}\end{sphinxVerbatimInput}
\begin{sphinxVerbatimOutput}

\begin{sphinxuseclass}{cell_output}
\begin{sphinxVerbatim}[commandchars=\\\{\}]
*** Reading ePS output file
/home/paul/github/ePSproc/data/photoionization/n2\PYGZus{}3sg\PYGZus{}0.1\PYGZhy{}50.1eV\PYGZus{}A2.inp.out
Found 102 sets of matrix elements
Read 102 sets of matrix elements (0 blank records)
Found 2 symmetries
    \PYGZsq{}SU\PYGZsq{}    \PYGZsq{}PU\PYGZsq{}

Found 51 energies
Found 2 atoms
Found 18 data records
Found 3 sets of cross sections
\end{sphinxVerbatim}

\end{sphinxuseclass}\end{sphinxVerbatimOutput}

\end{sphinxuseclass}
\begin{sphinxuseclass}{cell}\begin{sphinxVerbatimInput}

\begin{sphinxuseclass}{cell_input}
\begin{sphinxVerbatim}[commandchars=\\\{\}]
\PYG{c}{\PYGZpc{} Matrix elements are stored in a structure}
\PYG{n}{rlAll}
\end{sphinxVerbatim}

\end{sphinxuseclass}\end{sphinxVerbatimInput}
\begin{sphinxVerbatimOutput}

\begin{sphinxuseclass}{cell_output}
\begin{sphinxVerbatim}[commandchars=\\\{\}]
rlAll = 

  2x51 struct array with fields:

    eKE
    PE
    symm
    symmSet
    MbNorm
    rawIdyHead
    rawIdy1
    rlnlHead
    rlnl1
    rawIdy2
    rlnl2
    pWaveAll
\end{sphinxVerbatim}

\end{sphinxuseclass}\end{sphinxVerbatimOutput}

\end{sphinxuseclass}
\begin{sphinxuseclass}{cell}\begin{sphinxVerbatimInput}

\begin{sphinxuseclass}{cell_input}
\begin{sphinxVerbatim}[commandchars=\\\{\}]
\PYG{c}{\PYGZpc{} GetCro outputs (total cross\PYGZhy{}secions, LF betas)}
\PYG{n}{getCro}
\end{sphinxVerbatim}

\end{sphinxuseclass}\end{sphinxVerbatimInput}
\begin{sphinxVerbatimOutput}

\begin{sphinxuseclass}{cell_output}
\begin{sphinxVerbatim}[commandchars=\\\{\}]
getCro = 

  1x3 struct array with fields:

    GetCro
\end{sphinxVerbatim}

\end{sphinxuseclass}\end{sphinxVerbatimOutput}

\end{sphinxuseclass}
\begin{sphinxuseclass}{cell}\begin{sphinxVerbatimInput}

\begin{sphinxuseclass}{cell_input}
\begin{sphinxVerbatim}[commandchars=\\\{\}]
\PYG{c}{\PYGZpc{} General calculation params}
\PYG{n}{params}
\end{sphinxVerbatim}

\end{sphinxuseclass}\end{sphinxVerbatimInput}
\begin{sphinxVerbatimOutput}

\begin{sphinxuseclass}{cell_output}
\begin{sphinxVerbatim}[commandchars=\\\{\}]
params = 

  struct with fields:

        symmList: \PYGZob{}\PYGZsq{}SU\PYGZsq{}  \PYGZsq{}PU\PYGZsq{}  \PYGZsq{}All\PYGZsq{}\PYGZcb{}
             eKE: [1x51 double]
         symmAll: \PYGZob{}1x102 cell\PYGZcb{}
            eAll: [1x102 double]
        nRecords: 102
       nEnergies: 51
          nSymms: 2
           gLmax: 11
        blankRec: 0
        missingE: [1x0 double]
        pWaveAll: [144x51x3 double]
      pWaveAllMb: [144x51x3 double]
        LMallInd: [144x2 double]
          coords: \PYGZob{}1x5 cell\PYGZcb{}
     dataRecords: \PYGZob{}18x2 cell\PYGZcb{}
              IP: 15.5800
    GetCroHeader: \PYGZob{}\PYGZob{}1x1 cell\PYGZcb{}\PYGZcb{}
        fileBase: \PYGZsq{}/home/paul/github/ePSproc/data/photoionization/n2\PYGZus{}3sg\PYGZus{}0.1\PYGZhy{}50.1eV\PYGZus{}A2.inp.out\PYGZsq{}
        fileName: \PYGZsq{}n2\PYGZus{}3sg\PYGZus{}0.1\PYGZhy{}50.1eV\PYGZus{}A2.inp.out\PYGZsq{}
\end{sphinxVerbatim}

\end{sphinxuseclass}\end{sphinxVerbatimOutput}

\end{sphinxuseclass}

\subsection{Plot cross\sphinxhyphen{}sections and betas}
\label{\detokenize{testChpt/ePSproc_Matlab_demo_notebook_090821:plot-cross-sections-and-betas}}
\sphinxAtStartPar
These are taken from the \sphinxcode{\sphinxupquote{GetCro}} segments in the ePS output files, and correspond to results for an isotropic ensemble of molecules, i.e. observables in the lab frame (LF) for 1\sphinxhyphen{}photon ionization (see \sphinxhref{https://epsproc.readthedocs.io/en/latest/ePS\_ePSproc\_tutorial/ePS\_tutorial\_080520.html\#Theoretical-background}{the ePS tutorial for more details}).

\begin{sphinxuseclass}{cell}\begin{sphinxVerbatimInput}

\begin{sphinxuseclass}{cell_input}
\begin{sphinxVerbatim}[commandchars=\\\{\}]
\PYG{c}{\PYGZpc{}\PYGZpc{} Plot GetCro results for each symm \PYGZam{} total}

\PYG{n}{col}\PYG{p}{=}\PYG{l+m+mi}{2}\PYG{p}{;}\PYG{+w}{  }\PYG{c}{\PYGZpc{} Select column from getCro output (see params.GetCroHeader)}

\PYG{n+nb}{figure}\PYG{p}{(}\PYG{l+s}{\PYGZsq{}}\PYG{l+s}{color\PYGZsq{}}\PYG{p}{,}\PYG{p}{[}\PYG{l+m+mi}{1}\PYG{+w}{ }\PYG{l+m+mi}{1}\PYG{+w}{ }\PYG{l+m+mi}{1}\PYG{p}{]}\PYG{p}{,}\PYG{l+s}{\PYGZsq{}}\PYG{l+s}{name\PYGZsq{}}\PYG{p}{,}\PYG{l+s}{\PYGZsq{}}\PYG{l+s}{GetCro outputs\PYGZsq{}}\PYG{p}{)}\PYG{p}{;}

\PYG{k}{for}\PYG{+w}{ }\PYG{n}{n}\PYG{p}{=}\PYG{l+m+mi}{1}\PYG{p}{:}\PYG{n+nb}{length}\PYG{p}{(}\PYG{n}{getCro}\PYG{p}{)}
\PYG{+w}{    }\PYG{n+nb}{plot}\PYG{p}{(}\PYG{n}{getCro}\PYG{p}{(}\PYG{n}{n}\PYG{p}{)}\PYG{p}{.}\PYG{n}{GetCro}\PYG{p}{(}\PYG{p}{:}\PYG{p}{,}\PYG{l+m+mi}{1}\PYG{p}{)}\PYG{o}{\PYGZhy{}}\PYG{n}{params}\PYG{p}{.}\PYG{n}{IP}\PYG{p}{,}\PYG{n}{getCro}\PYG{p}{(}\PYG{n}{n}\PYG{p}{)}\PYG{p}{.}\PYG{n}{GetCro}\PYG{p}{(}\PYG{p}{:}\PYG{p}{,}\PYG{n}{col}\PYG{p}{)}\PYG{p}{)}\PYG{p}{;}
\PYG{+w}{    }\PYG{n}{hold}\PYG{+w}{ }\PYG{l+s}{on}\PYG{p}{;}
\PYG{k}{end}

\PYG{n+nb}{title}\PYG{p}{(}\PYG{p}{\PYGZob{}}\PYG{p}{[}\PYG{l+s}{\PYGZsq{}}\PYG{l+s}{NO\PYGZus{}2 ePS resutls, files \PYGZsq{}}\PYG{+w}{ }\PYG{n+nb}{strrep}\PYG{p}{(}\PYG{n+nb}{fileName}\PYG{p}{,}\PYG{l+s}{\PYGZsq{}}\PYG{l+s}{\PYGZus{}\PYGZsq{}}\PYG{p}{,}\PYG{l+s}{\PYGZsq{}}\PYG{l+s}{\PYGZbs{}\PYGZus{}\PYGZsq{}}\PYG{p}{)}\PYG{p}{]}\PYG{p}{;}\PYG{+w}{ }\PYG{l+s}{\PYGZsq{}}\PYG{l+s}{X\PYGZhy{}sects from ePS(GetCro) results\PYGZsq{}}\PYG{p}{\PYGZcb{}}\PYG{p}{)}\PYG{p}{;}
\PYG{n+nb}{xlabel}\PYG{p}{(}\PYG{l+s}{\PYGZsq{}}\PYG{l+s}{eKE/eV\PYGZsq{}}\PYG{p}{)}\PYG{p}{;}
\PYG{n+nb}{ylabel}\PYG{p}{(}\PYG{l+s}{\PYGZsq{}}\PYG{l+s}{X\PYGZhy{}sect/Mb\PYGZsq{}}\PYG{p}{)}\PYG{p}{;}

\PYG{n+nb}{legend}\PYG{p}{(}\PYG{p}{[}\PYG{n}{params}\PYG{p}{.}\PYG{n}{symmList}\PYG{+w}{ }\PYG{l+s}{\PYGZsq{}}\PYG{l+s}{Sum\PYGZsq{}}\PYG{p}{]}\PYG{p}{)}\PYG{p}{;}
\end{sphinxVerbatim}

\end{sphinxuseclass}\end{sphinxVerbatimInput}
\begin{sphinxVerbatimOutput}

\begin{sphinxuseclass}{cell_output}
\begin{sphinxVerbatim}[commandchars=\\\{\}]
Warning: MATLAB has disabled some advanced graphics rendering features by switching to software OpenGL. For more information, click \PYGZlt{}a href=\PYGZdq{}matlab:opengl(\PYGZsq{}problems\PYGZsq{})\PYGZdq{}\PYGZgt{}here\PYGZlt{}/a\PYGZgt{}.
Warning: Ignoring extra legend entries.
\PYGZgt{} In legend\PYGZgt{}set\PYGZus{}children\PYGZus{}and\PYGZus{}strings (line 646)
  In legend\PYGZgt{}make\PYGZus{}legend (line 316)
  In legend (line 259)
\end{sphinxVerbatim}

\noindent\sphinxincludegraphics{{ePSproc_Matlab_demo_notebook_090821_10_1}.png}

\end{sphinxuseclass}\end{sphinxVerbatimOutput}

\end{sphinxuseclass}

\subsection{MFPADs}
\label{\detokenize{testChpt/ePSproc_Matlab_demo_notebook_090821:mfpads}}
\sphinxAtStartPar
These are calculated numerically from the matrix elements, for a given polarization geometry and symmetry (the \sphinxhref{https://epsproc.readthedocs.io/en/dev/demos/ePSproc\_demo\_Aug2019.html\#Calculate-MFPADs}{method is the same as the python version of the routine}).

\begin{sphinxuseclass}{cell}\begin{sphinxVerbatimInput}

\begin{sphinxuseclass}{cell_input}
\begin{sphinxVerbatim}[commandchars=\\\{\}]
\PYG{c}{\PYGZpc{}\PYGZpc{} *** Calculate MFPADs \PYGZhy{} single polarization geometry, all energies and symmetries}
\PYG{c}{\PYGZpc{}  Calculate for specified Euler angles (polarization geometry) \PYGZam{} energies}

\PYG{c}{\PYGZpc{} Set resolution for calculated I(theta,phi) surfaces}
\PYG{n}{res}\PYG{p}{=}\PYG{l+m+mi}{100}\PYG{p}{;}

\PYG{c}{\PYGZpc{} ip components to use from ePS output (1=length gauge, 2=velocity gauge)}
\PYG{n}{ipComponents}\PYG{p}{=}\PYG{l+m+mi}{1}\PYG{p}{;}

\PYG{c}{\PYGZpc{} it components to use from ePS output (for degenerate cases), set an array here for as many components as required, e.g. it=1, it=[1 2] etc.}
\PYG{n}{it}\PYG{p}{=}\PYG{l+m+mi}{1}\PYG{p}{;}

\PYG{c}{\PYGZpc{} Set light polarization and axis rotations LF \PYGZhy{}\PYGZgt{} MF}
\PYG{n}{p}\PYG{p}{=}\PYG{l+m+mi}{0}\PYG{p}{;}\PYG{+w}{                }\PYG{c}{\PYGZpc{} p=0 for linearly pol. light, +/\PYGZhy{}1 for L/R circ. pol.}
\PYG{n}{eAngs}\PYG{p}{=}\PYG{p}{[}\PYG{l+m+mi}{0}\PYG{+w}{ }\PYG{l+m+mi}{0}\PYG{+w}{ }\PYG{l+m+mi}{0}\PYG{p}{]}\PYG{p}{;}\PYG{+w}{      }\PYG{c}{\PYGZpc{} Eugler angles for rotation of LF\PYGZhy{}\PYGZgt{}MF, set as [0 0 0] for z\PYGZhy{}pol, [0 pi/2 0] for x\PYGZhy{}pol, [pi/2 pi/2 0] for y\PYGZhy{}pol}
\PYG{n}{polLabel}\PYG{p}{=}\PYG{l+s}{\PYGZsq{}}\PYG{l+s}{z\PYGZsq{}}\PYG{p}{;}

\PYG{c}{\PYGZpc{} Run calculation \PYGZhy{} outputs are D, full set of MFPADs (summed over symmetries); Xsect, calculated X\PYGZhy{}sects; calcsAll, structure with results for all symmetries.}
\PYG{p}{[}\PYG{n}{Xsect}\PYG{p}{,}\PYG{+w}{ }\PYG{n}{calcsAll}\PYG{p}{,}\PYG{+w}{ }\PYG{n}{pWaves}\PYG{p}{]}\PYG{p}{=}\PYG{n}{ePSproc\PYGZus{}MFPAD}\PYG{p}{(}\PYG{n}{rlAll}\PYG{p}{,}\PYG{n}{p}\PYG{p}{,}\PYG{n}{eAngs}\PYG{p}{,}\PYG{n}{it}\PYG{p}{,}\PYG{n}{ipComponents}\PYG{p}{,}\PYG{n}{res}\PYG{p}{)}\PYG{p}{;}

\PYG{c}{\PYGZpc{} Add pol labels \PYGZhy{} currently expected in plotting routine, but not set in MFPAD routine}
\PYG{k}{for}\PYG{+w}{ }\PYG{n}{n}\PYG{p}{=}\PYG{l+m+mi}{1}\PYG{p}{:}\PYG{n+nb}{size}\PYG{p}{(}\PYG{n}{calcsAll}\PYG{p}{,}\PYG{l+m+mi}{2}\PYG{p}{)}
\PYG{+w}{    }\PYG{k}{for}\PYG{+w}{ }\PYG{n}{symmIn}\PYG{p}{=}\PYG{l+m+mi}{1}\PYG{p}{:}\PYG{n+nb}{size}\PYG{p}{(}\PYG{n}{calcsAll}\PYG{p}{,}\PYG{l+m+mi}{1}\PYG{p}{)}
\PYG{+w}{        }\PYG{n}{calcsAll}\PYG{p}{(}\PYG{n}{symmInd}\PYG{p}{,}\PYG{n}{n}\PYG{p}{)}\PYG{p}{.}\PYG{n}{polLabel}\PYG{p}{=}\PYG{n}{polLabel}\PYG{p}{;}
\PYG{+w}{    }\PYG{k}{end}
\PYG{k}{end}
\PYG{+w}{        }
\end{sphinxVerbatim}

\end{sphinxuseclass}\end{sphinxVerbatimInput}

\end{sphinxuseclass}
\begin{sphinxuseclass}{cell}\begin{sphinxVerbatimInput}

\begin{sphinxuseclass}{cell_input}
\begin{sphinxVerbatim}[commandchars=\\\{\}]
\PYG{c}{\PYGZpc{} Results are output as a structure, dims (symmetries, energies).}
\PYG{n}{calcsAll}
\end{sphinxVerbatim}

\end{sphinxuseclass}\end{sphinxVerbatimInput}
\begin{sphinxVerbatimOutput}

\begin{sphinxuseclass}{cell_output}
\begin{sphinxVerbatim}[commandchars=\\\{\}]
calcsAll = 

  3x51 struct array with fields:

    D
    C
    Cthres
    eKE
    symm
    euler
    Xsect
    XsectD
    Rlf
    p
    Cind
\end{sphinxVerbatim}

\end{sphinxuseclass}\end{sphinxVerbatimOutput}

\end{sphinxuseclass}
\begin{sphinxuseclass}{cell}\begin{sphinxVerbatimInput}

\begin{sphinxuseclass}{cell_input}
\begin{sphinxVerbatim}[commandchars=\\\{\}]
\PYG{c}{\PYGZpc{}plot \PYGZhy{}s 800,400}
\end{sphinxVerbatim}

\end{sphinxuseclass}\end{sphinxVerbatimInput}

\end{sphinxuseclass}
\sphinxAtStartPar
(Above is a line\sphinxhyphen{}magic for setting displayed plot size with the Calysto kernel \sphinxhyphen{} see \sphinxhref{https://nbviewer.jupyter.org/github/Calysto/matlab\_kernel/blob/master/matlab\_kernel.ipynb}{the demo notebook} for more.)

\begin{sphinxuseclass}{cell}\begin{sphinxVerbatimInput}

\begin{sphinxuseclass}{cell_input}
\begin{sphinxVerbatim}[commandchars=\\\{\}]
\PYG{c}{\PYGZpc{}\PYGZpc{} Plotting \PYGZhy{} MFPAD panel plots}

\PYG{c}{\PYGZpc{} Set plot ranges}
\PYG{n}{symmInd}\PYG{p}{=}\PYG{l+m+mi}{1}\PYG{p}{;}\PYG{+w}{     }\PYG{c}{\PYGZpc{} Select symmetry (by index into calcsAll rows). Final symmetry state is set as sum over all symmetries}
\PYG{c}{\PYGZpc{} eRange=1;      \PYGZpc{} Select energies (by index into calcsAll cols)}
\PYG{n}{eRange}\PYG{p}{=}\PYG{l+m+mi}{1}\PYG{p}{:}\PYG{l+m+mi}{3}\PYG{p}{:}\PYG{l+m+mi}{20}\PYG{p}{;}

\PYG{c}{\PYGZpc{} Additional options (optional)}
\PYG{n}{sPlotSet}\PYG{p}{=}\PYG{p}{[}\PYG{l+m+mi}{2}\PYG{+w}{ }\PYG{l+m+mi}{4}\PYG{p}{]}\PYG{p}{;}\PYG{+w}{             }\PYG{c}{\PYGZpc{} Set [rows cols] for subplot panels. The final panel will be replaced with a diagram of the geometry}
\PYG{c}{\PYGZpc{} titlePrefix=\PYGZsq{}NO2 testing\PYGZsq{};  \PYGZpc{} Set a title prefix for the figure}
\PYG{n}{titlePrefix}\PYG{p}{=}\PYG{l+s}{\PYGZsq{}}\PYG{l+s}{\PYGZsq{}}\PYG{p}{;}

\PYG{n}{ePSproc\PYGZus{}MFPAD\PYGZus{}plot}\PYG{p}{(}\PYG{n}{calcsAll}\PYG{p}{,}\PYG{n}{eRange}\PYG{p}{,}\PYG{n}{symmInd}\PYG{p}{,}\PYG{n}{params}\PYG{p}{,}\PYG{n}{sPlotSet}\PYG{p}{,}\PYG{n}{titlePrefix}\PYG{p}{)}\PYG{p}{;}
\PYG{c}{\PYGZpc{} ePSproc\PYGZus{}MFPAD\PYGZus{}plot(calcsAll,eRange,symmInd,params,sPlotSet,\PYGZsq{}\PYGZsq{},\PYGZsq{}n\PYGZsq{},\PYGZsq{}off\PYGZsq{});}
\PYG{c}{\PYGZpc{} ePSproc\PYGZus{}MFPAD\PYGZus{}plot(calcsAll,eRange,symmInd,params,[2 4],\PYGZsq{}\PYGZsq{},\PYGZsq{}n\PYGZsq{},\PYGZsq{}off\PYGZsq{});}
\end{sphinxVerbatim}

\end{sphinxuseclass}\end{sphinxVerbatimInput}
\begin{sphinxVerbatimOutput}

\begin{sphinxuseclass}{cell_output}
\noindent\sphinxincludegraphics{{ePSproc_Matlab_demo_notebook_090821_16_0}.png}

\end{sphinxuseclass}\end{sphinxVerbatimOutput}

\end{sphinxuseclass}
\begin{sphinxuseclass}{cell}\begin{sphinxVerbatimInput}

\begin{sphinxuseclass}{cell_input}
\begin{sphinxVerbatim}[commandchars=\\\{\}]
\PYG{c}{\PYGZpc{} Calculate \PYGZam{} plot for a different polarization state}
\PYG{n}{eAngs}\PYG{+w}{ }\PYG{p}{=}\PYG{+w}{ }\PYG{p}{[}\PYG{l+m+mi}{0}\PYG{+w}{ }\PYG{n+nb}{pi}\PYG{o}{/}\PYG{l+m+mi}{2}\PYG{+w}{ }\PYG{l+m+mi}{0}\PYG{p}{]}\PYG{p}{;}\PYG{+w}{  }\PYG{c}{\PYGZpc{} x\PYGZhy{}pol case}
\PYG{n}{polLabel}\PYG{+w}{ }\PYG{p}{=}\PYG{+w}{ }\PYG{l+s}{\PYGZsq{}}\PYG{l+s}{x\PYGZsq{}}\PYG{p}{;}

\PYG{p}{[}\PYG{n}{Xsect}\PYG{p}{,}\PYG{+w}{ }\PYG{n}{calcsAll}\PYG{p}{,}\PYG{+w}{ }\PYG{n}{pWaves}\PYG{p}{]}\PYG{p}{=}\PYG{n}{ePSproc\PYGZus{}MFPAD}\PYG{p}{(}\PYG{n}{rlAll}\PYG{p}{,}\PYG{n}{p}\PYG{p}{,}\PYG{n}{eAngs}\PYG{p}{,}\PYG{n}{it}\PYG{p}{,}\PYG{n}{ipComponents}\PYG{p}{,}\PYG{n}{res}\PYG{p}{)}\PYG{p}{;}

\PYG{c}{\PYGZpc{} Add pol labels \PYGZhy{} currently expected in plotting routine, but not set in MFPAD routine}
\PYG{k}{for}\PYG{+w}{ }\PYG{n}{n}\PYG{p}{=}\PYG{l+m+mi}{1}\PYG{p}{:}\PYG{n+nb}{size}\PYG{p}{(}\PYG{n}{calcsAll}\PYG{p}{,}\PYG{l+m+mi}{2}\PYG{p}{)}
\PYG{+w}{    }\PYG{k}{for}\PYG{+w}{ }\PYG{n}{symmIn}\PYG{p}{=}\PYG{l+m+mi}{1}\PYG{p}{:}\PYG{n+nb}{size}\PYG{p}{(}\PYG{n}{calcsAll}\PYG{p}{,}\PYG{l+m+mi}{1}\PYG{p}{)}
\PYG{+w}{        }\PYG{n}{calcsAll}\PYG{p}{(}\PYG{n}{symmInd}\PYG{p}{,}\PYG{n}{n}\PYG{p}{)}\PYG{p}{.}\PYG{n}{polLabel}\PYG{p}{=}\PYG{n}{polLabel}\PYG{p}{;}
\PYG{+w}{    }\PYG{k}{end}
\PYG{k}{end}

\PYG{n}{symmInd}\PYG{p}{=}\PYG{l+m+mi}{3}\PYG{p}{;}
\PYG{n}{ePSproc\PYGZus{}MFPAD\PYGZus{}plot}\PYG{p}{(}\PYG{n}{calcsAll}\PYG{p}{,}\PYG{n}{eRange}\PYG{p}{,}\PYG{n}{symmInd}\PYG{p}{,}\PYG{n}{params}\PYG{p}{,}\PYG{n}{sPlotSet}\PYG{p}{,}\PYG{n}{titlePrefix}\PYG{p}{)}\PYG{p}{;}
\end{sphinxVerbatim}

\end{sphinxuseclass}\end{sphinxVerbatimInput}
\begin{sphinxVerbatimOutput}

\begin{sphinxuseclass}{cell_output}
\begin{sphinxVerbatim}[commandchars=\\\{\}]
symmInd =

     3
\end{sphinxVerbatim}

\noindent\sphinxincludegraphics{{ePSproc_Matlab_demo_notebook_090821_17_1}.png}

\end{sphinxuseclass}\end{sphinxVerbatimOutput}

\end{sphinxuseclass}
\begin{sphinxuseclass}{cell}\begin{sphinxVerbatimInput}

\begin{sphinxuseclass}{cell_input}
\begin{sphinxVerbatim}[commandchars=\\\{\}]
\PYG{c}{\PYGZpc{} Calculate \PYGZam{} plot for a different polarization state}
\PYG{n}{eAngs}\PYG{+w}{ }\PYG{p}{=}\PYG{+w}{ }\PYG{p}{[}\PYG{l+m+mi}{0}\PYG{+w}{ }\PYG{n+nb}{pi}\PYG{o}{/}\PYG{l+m+mi}{4}\PYG{+w}{ }\PYG{l+m+mi}{0}\PYG{p}{]}\PYG{p}{;}\PYG{+w}{  }\PYG{c}{\PYGZpc{} Diagonal pol case}
\PYG{n}{polLabel}\PYG{+w}{ }\PYG{p}{=}\PYG{+w}{ }\PYG{l+s}{\PYGZsq{}}\PYG{l+s}{x\PYGZsq{}}\PYG{p}{;}

\PYG{p}{[}\PYG{n}{Xsect}\PYG{p}{,}\PYG{+w}{ }\PYG{n}{calcsAll}\PYG{p}{,}\PYG{+w}{ }\PYG{n}{pWaves}\PYG{p}{]}\PYG{p}{=}\PYG{n}{ePSproc\PYGZus{}MFPAD}\PYG{p}{(}\PYG{n}{rlAll}\PYG{p}{,}\PYG{n}{p}\PYG{p}{,}\PYG{n}{eAngs}\PYG{p}{,}\PYG{n}{it}\PYG{p}{,}\PYG{n}{ipComponents}\PYG{p}{,}\PYG{n}{res}\PYG{p}{)}\PYG{p}{;}

\PYG{c}{\PYGZpc{} Add pol labels \PYGZhy{} currently expected in plotting routine, but not set in MFPAD routine}
\PYG{k}{for}\PYG{+w}{ }\PYG{n}{n}\PYG{p}{=}\PYG{l+m+mi}{1}\PYG{p}{:}\PYG{n+nb}{size}\PYG{p}{(}\PYG{n}{calcsAll}\PYG{p}{,}\PYG{l+m+mi}{2}\PYG{p}{)}
\PYG{+w}{    }\PYG{k}{for}\PYG{+w}{ }\PYG{n}{symmIn}\PYG{p}{=}\PYG{l+m+mi}{1}\PYG{p}{:}\PYG{n+nb}{size}\PYG{p}{(}\PYG{n}{calcsAll}\PYG{p}{,}\PYG{l+m+mi}{1}\PYG{p}{)}
\PYG{+w}{        }\PYG{n}{calcsAll}\PYG{p}{(}\PYG{n}{symmInd}\PYG{p}{,}\PYG{n}{n}\PYG{p}{)}\PYG{p}{.}\PYG{n}{polLabel}\PYG{p}{=}\PYG{n}{polLabel}\PYG{p}{;}
\PYG{+w}{    }\PYG{k}{end}
\PYG{k}{end}

\PYG{c}{\PYGZpc{} Plot all symmetries}
\PYG{k}{for}\PYG{+w}{ }\PYG{n}{symmInd}\PYG{+w}{ }\PYG{p}{=}\PYG{+w}{ }\PYG{l+m+mi}{1}\PYG{p}{:}\PYG{l+m+mi}{3}
\PYG{+w}{    }\PYG{n}{ePSproc\PYGZus{}MFPAD\PYGZus{}plot}\PYG{p}{(}\PYG{n}{calcsAll}\PYG{p}{,}\PYG{n}{eRange}\PYG{p}{,}\PYG{n}{symmInd}\PYG{p}{,}\PYG{n}{params}\PYG{p}{,}\PYG{n}{sPlotSet}\PYG{p}{,}\PYG{n}{titlePrefix}\PYG{p}{)}\PYG{p}{;}
\PYG{+w}{    }
\PYG{k}{end}
\end{sphinxVerbatim}

\end{sphinxuseclass}\end{sphinxVerbatimInput}
\begin{sphinxVerbatimOutput}

\begin{sphinxuseclass}{cell_output}
\noindent\sphinxincludegraphics{{ePSproc_Matlab_demo_notebook_090821_18_0}.png}

\noindent\sphinxincludegraphics{{ePSproc_Matlab_demo_notebook_090821_18_1}.png}

\noindent\sphinxincludegraphics{{ePSproc_Matlab_demo_notebook_090821_18_2}.png}

\end{sphinxuseclass}\end{sphinxVerbatimOutput}

\end{sphinxuseclass}

\subsection{MF \protect\(\beta_{LM}\protect\)}
\label{\detokenize{testChpt/ePSproc_Matlab_demo_notebook_090821:mf-beta-lm}}
\begin{sphinxuseclass}{cell}\begin{sphinxVerbatimInput}

\begin{sphinxuseclass}{cell_input}
\begin{sphinxVerbatim}[commandchars=\\\{\}]
\PYG{c}{\PYGZpc{}\PYGZpc{} *** Calculate MFPADs \PYGZhy{} single polarization geometry, all energies and symmetries}
\PYG{c}{\PYGZpc{}  Calculate for specified Euler angles (polarization geometry) \PYGZam{} energies}

\PYG{c}{\PYGZpc{} Set resolution for calculated I(theta,phi) surfaces}
\PYG{n}{res}\PYG{p}{=}\PYG{l+m+mi}{100}\PYG{p}{;}

\PYG{c}{\PYGZpc{} ip components to use from ePS output (1=length gauge, 2=velocity gauge)}
\PYG{n}{ipComponents}\PYG{p}{=}\PYG{l+m+mi}{1}\PYG{p}{;}

\PYG{c}{\PYGZpc{} it components to use from ePS output (for degenerate cases), set an array here for as many components as required, e.g. it=1, it=[1 2] etc.}
\PYG{n}{it}\PYG{p}{=}\PYG{l+m+mi}{1}\PYG{p}{;}

\PYG{c}{\PYGZpc{} Set light polarization and axis rotations LF \PYGZhy{}\PYGZgt{} MF}
\PYG{n}{p}\PYG{p}{=}\PYG{l+m+mi}{0}\PYG{p}{;}\PYG{+w}{                }\PYG{c}{\PYGZpc{} p=0 for linearly pol. light, +/\PYGZhy{}1 for L/R circ. pol.}
\PYG{n}{eAngs}\PYG{p}{=}\PYG{p}{[}\PYG{l+m+mi}{0}\PYG{+w}{ }\PYG{l+m+mi}{0}\PYG{+w}{ }\PYG{l+m+mi}{0}\PYG{p}{]}\PYG{p}{;}\PYG{+w}{      }\PYG{c}{\PYGZpc{} Eugler angles for rotation of LF\PYGZhy{}\PYGZgt{}MF, set as [0 0 0] for z\PYGZhy{}pol, [0 pi/2 0] for x\PYGZhy{}pol, [pi/2 pi/2 0] for y\PYGZhy{}pol}
\PYG{n}{polLabel}\PYG{p}{=}\PYG{l+s}{\PYGZsq{}}\PYG{l+s}{z\PYGZsq{}}\PYG{p}{;}

\PYG{c}{\PYGZpc{} Run calculation \PYGZhy{} outputs are D, full set of MFPADs (summed over symmetries); Xsect, calculated X\PYGZhy{}sects; calcsAll, structure with results for all symmetries.}
\PYG{n}{calcsAll}\PYG{p}{=}\PYG{n}{ePSproc\PYGZus{}MFPAD}\PYG{p}{(}\PYG{n}{rlAll}\PYG{p}{,}\PYG{n}{p}\PYG{p}{,}\PYG{n}{eAngs}\PYG{p}{,}\PYG{n}{it}\PYG{p}{,}\PYG{n}{ipComponents}\PYG{p}{,}\PYG{n}{res}\PYG{p}{)}\PYG{p}{;}

\PYG{c}{\PYGZpc{} Add pol labels \PYGZhy{} currently expected in plotting routine, but not set in MFPAD routine}
\PYG{c}{\PYGZpc{} for n=1:size(calcsAll,2)}
\PYG{c}{\PYGZpc{}     for symmIn=1:size(calcsAll,1)}
\PYG{c}{\PYGZpc{}         calcsAll(symmInd,n).polLabel=polLabel;}
\PYG{c}{\PYGZpc{}     end}
\PYG{c}{\PYGZpc{} end}
\PYG{+w}{        }
\end{sphinxVerbatim}

\end{sphinxuseclass}\end{sphinxVerbatimInput}

\end{sphinxuseclass}
\begin{sphinxuseclass}{cell}\begin{sphinxVerbatimInput}

\begin{sphinxuseclass}{cell_input}
\begin{sphinxVerbatim}[commandchars=\\\{\}]
\PYG{n+nb}{plot}\PYG{p}{(}\PYG{n}{calcsAll}\PYG{p}{.}\PYG{o}{\PYGZsq{}}\PYG{p}{)}
\end{sphinxVerbatim}

\end{sphinxuseclass}\end{sphinxVerbatimInput}
\begin{sphinxVerbatimOutput}

\begin{sphinxuseclass}{cell_output}
\noindent\sphinxincludegraphics{{ePSproc_Matlab_demo_notebook_090821_21_0}.png}

\end{sphinxuseclass}\end{sphinxVerbatimOutput}

\end{sphinxuseclass}






\renewcommand{\indexname}{Index}
\printindex
\end{document}