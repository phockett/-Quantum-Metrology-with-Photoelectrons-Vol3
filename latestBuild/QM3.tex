%% Generated by Sphinx.
\def\sphinxdocclass{jupyterBook}
\documentclass[letterpaper,10pt,english]{jupyterBook}
\ifdefined\pdfpxdimen
   \let\sphinxpxdimen\pdfpxdimen\else\newdimen\sphinxpxdimen
\fi \sphinxpxdimen=.75bp\relax
\ifdefined\pdfimageresolution
    \pdfimageresolution= \numexpr \dimexpr1in\relax/\sphinxpxdimen\relax
\fi
%% let collapsible pdf bookmarks panel have high depth per default
\PassOptionsToPackage{bookmarksdepth=5}{hyperref}
%% turn off hyperref patch of \index as sphinx.xdy xindy module takes care of
%% suitable \hyperpage mark-up, working around hyperref-xindy incompatibility
\PassOptionsToPackage{hyperindex=false}{hyperref}
%% memoir class requires extra handling
\makeatletter\@ifclassloaded{memoir}
{\ifdefined\memhyperindexfalse\memhyperindexfalse\fi}{}\makeatother

\PassOptionsToPackage{warn}{textcomp}

\catcode`^^^^00a0\active\protected\def^^^^00a0{\leavevmode\nobreak\ }
\usepackage{cmap}
\usepackage{fontspec}
\defaultfontfeatures[\rmfamily,\sffamily,\ttfamily]{}
\usepackage{amsmath,amssymb,amstext}
\usepackage{polyglossia}
\setmainlanguage{english}



\setmainfont{FreeSerif}[
  Extension      = .otf,
  UprightFont    = *,
  ItalicFont     = *Italic,
  BoldFont       = *Bold,
  BoldItalicFont = *BoldItalic
]
\setsansfont{FreeSans}[
  Extension      = .otf,
  UprightFont    = *,
  ItalicFont     = *Oblique,
  BoldFont       = *Bold,
  BoldItalicFont = *BoldOblique,
]
\setmonofont{FreeMono}[
  Extension      = .otf,
  UprightFont    = *,
  ItalicFont     = *Oblique,
  BoldFont       = *Bold,
  BoldItalicFont = *BoldOblique,
]



\usepackage[Bjarne]{fncychap}
\usepackage[,numfigreset=1,mathnumfig]{sphinx}

\fvset{fontsize=\small}
\usepackage{geometry}


% Include hyperref last.
\usepackage{hyperref}
% Fix anchor placement for figures with captions.
\usepackage{hypcap}% it must be loaded after hyperref.
% Set up styles of URL: it should be placed after hyperref.
\urlstyle{same}

\addto\captionsenglish{\renewcommand{\contentsname}{Frontmatter}}

\usepackage{sphinxmessages}



        % Start of preamble defined in sphinx-jupyterbook-latex %
         \usepackage[Latin,Greek]{ucharclasses}
        \usepackage{unicode-math}
        % fixing title of the toc
        \addto\captionsenglish{\renewcommand{\contentsname}{Contents}}
        \hypersetup{
            pdfencoding=auto,
            psdextra
        }
        % End of preamble defined in sphinx-jupyterbook-latex %
        

\title{Quantum Metrology with Photoelectrons Vol. 3 *Analysis methodologies*}
\date{Nov 21, 2022}
\release{}
\author{Paul Hockett}
\newcommand{\sphinxlogo}{\vbox{}}
\renewcommand{\releasename}{}
\makeindex
\begin{document}

\pagestyle{empty}
\sphinxmaketitle
\pagestyle{plain}
\sphinxtableofcontents
\pagestyle{normal}
\phantomsection\label{\detokenize{intro::doc}}


\sphinxAtStartPar
Quantum Metrology with Photoelectrons Volume 3: \sphinxstyleemphasis{Analysis
methodologies}, an open source executable book. This repository contains the source documents (mainly Jupyter Notebooks in Python) and notes for the book, as of Jan 2022 writing is in progress, and the \sphinxhref{https://phockett.github.io/Quantum-Metrology-with-Photoelectrons-Vol3/}{current HTML build can be found online}. The book is due to be finished in 2023, and will be published by IOP Press \sphinxhyphen{} see below for more details.

\begin{DUlineblock}{0em}
\item[] \sphinxstylestrong{\Large Series abstract}
\end{DUlineblock}

\sphinxAtStartPar
Photoionization is an interferometric process, in which multiple paths can contribute to the final continuum photoelectron wavefunction. At the simplest level, interferences between different final angular momentum states are manifest in the energy and angle resolved photoelectron spectra: metrology schemes making use of these interferograms are thus phase\sphinxhyphen{}sensitive, and provide a powerful route to detailed understanding of photoionization. In these cases, the continuum wavefunction (and underlying scattering dynamics) can be characterised. At a more complex level, such measurements can also provide a powerful probe for other processes of interest, leading to a more general class of quantum metrology built on phase\sphinxhyphen{}sensitive photoelectron imaging.  Since the turn of the century, the increasing availability of photoelectron imaging experiments, along with the increasing sophistication of experimental techniques, and the availability of computational resources for analysis and numerics, has allowed for significant developments in such photoelectron metrology.

\begin{DUlineblock}{0em}
\item[] \sphinxstylestrong{\large About the books}
\end{DUlineblock}

\sphinxAtStartPar
\sphinxincludegraphics{{mock_covers_2vol_020318}.png}
\begin{itemize}
\item {} 
\sphinxAtStartPar
Volume I covers the core physics of photoionization, including a range of computational examples. The material is presented as both reference and tutorial, and should appeal to readers of all levels. ISBN 978\sphinxhyphen{}1\sphinxhyphen{}6817\sphinxhyphen{}4684\sphinxhyphen{}5, \sphinxurl{http://iopscience.iop.org/book/978-1-6817-4684-5} (IOP Press, 2018)

\item {} 
\sphinxAtStartPar
Volume II explores applications, and the development of quantum metrology schemes based on photoelectron measurements. The material is more technical, and will appeal more to the specialist reader. ISBN 978\sphinxhyphen{}1\sphinxhyphen{}6817\sphinxhyphen{}4688\sphinxhyphen{}3, \sphinxurl{http://iopscience.iop.org/book/978-1-6817-4688-3} (IOP Press, 2018)

\end{itemize}

\sphinxAtStartPar
Additional online resources for Vols. I \& II can be found on \sphinxhref{https://osf.io/q2v3g/wiki/home/}{OSF} and \sphinxhref{https://github.com/phockett/Quantum-Metrology-with-Photoelectrons}{Github}.
\begin{itemize}
\item {} 
\sphinxAtStartPar
Volume III in the series will continue this exploration, with a focus on numerical analysis techniques, forging a closer link between experimental and theoretical results, and making the methodologies discussed directly accessible via new software. The book is due for publication by IOP due in 2023; this volume is also open\sphinxhyphen{}source, with a live HTML version at \sphinxurl{https://phockett.github.io/Quantum-Metrology-with-Photoelectrons-Vol3/} and source available at \sphinxurl{https://github.com/phockett/Quantum-Metrology-with-Photoelectrons-Vol3}.

\end{itemize}

\sphinxAtStartPar
For some additional details and motivations (including topical video), see \sphinxhref{https://phockett.github.io/ePSdata/about.html\#Motivation}{the ePSdata project}.

\begin{DUlineblock}{0em}
\item[] \sphinxstylestrong{\large Technical details}
\end{DUlineblock}

\sphinxAtStartPar
This repository contains:
\begin{itemize}
\item {} 
\sphinxAtStartPar
\sphinxcode{\sphinxupquote{doc\sphinxhyphen{}source}}: the source documents (mainly Jupyter Notebooks in Python)

\item {} 
\sphinxAtStartPar
\sphinxcode{\sphinxupquote{notes}}: additional notes for the book,

\item {} 
\sphinxAtStartPar
the \sphinxcode{\sphinxupquote{gh\sphinxhyphen{}pages}} branch contains the current HTML build, also available at \sphinxurl{https://phockett.github.io/Quantum-Metrology-with-Photoelectrons-Vol3/}

\end{itemize}

\sphinxAtStartPar
The project has been setup to use the \sphinxhref{https://jupyterbook.org/}{Jupyter Book} build\sphinxhyphen{}chain (which uses Sphinx on the back\sphinxhyphen{}end) to generate HTML and Latex outputs for publication from source Jupyter notebooks \& markdown files.

\sphinxAtStartPar
The work \sphinxstyleemphasis{within} the book will make use of the \sphinxhref{https://pemtk.readthedocs.io/en/latest/about.html}{Photoelectron Metrology Toolkit} platform for working with experimental \& theoretical data.

\sphinxAtStartPar
\sphinxincludegraphics{{ccdcf3feb912fb992ab79da89d86a2521bfe1c21}.png}

\begin{DUlineblock}{0em}
\item[] \sphinxstylestrong{\large Running code examples}
\end{DUlineblock}

\sphinxAtStartPar
Each Jupyter notebook (\sphinxcode{\sphinxupquote{*.ipynb}}) can be treated as a stand\sphinxhyphen{}alone computational document. These can be run/used/modified independently with an appropriately setup python environment (details to follow).

\begin{DUlineblock}{0em}
\item[] \sphinxstylestrong{\large Building the book}
\end{DUlineblock}

\sphinxAtStartPar
The full book can also be built from source:
\begin{enumerate}
\sphinxsetlistlabels{\arabic}{enumi}{enumii}{}{.}%
\item {} 
\sphinxAtStartPar
Clone this repository

\item {} 
\sphinxAtStartPar
Run \sphinxcode{\sphinxupquote{pip install \sphinxhyphen{}r requirements.txt}} (it is recommended you do this within a virtual environment)

\item {} 
\sphinxAtStartPar
(Optional) Edit the books source files located in the \sphinxcode{\sphinxupquote{doc\sphinxhyphen{}source/}} directory

\item {} 
\sphinxAtStartPar
Run \sphinxcode{\sphinxupquote{jupyter\sphinxhyphen{}book clean doc\sphinxhyphen{}source/}} to remove any existing builds

\item {} 
\sphinxAtStartPar
For an HTML build:
\begin{itemize}
\item {} 
\sphinxAtStartPar
Run \sphinxcode{\sphinxupquote{jupyter\sphinxhyphen{}book build doc\sphinxhyphen{}source/}}

\item {} 
\sphinxAtStartPar
A fully\sphinxhyphen{}rendered HTML version of the book will be built in \sphinxcode{\sphinxupquote{doc\sphinxhyphen{}source/\_build/html/}}.

\end{itemize}

\item {} 
\sphinxAtStartPar
For a LaTex \& PDF build:
\begin{itemize}
\item {} 
\sphinxAtStartPar
Run \sphinxcode{\sphinxupquote{jupyter\sphinxhyphen{}book build doc\sphinxhyphen{}source/ \sphinxhyphen{}\sphinxhyphen{}builder pdflatex}}

\item {} 
\sphinxAtStartPar
A fully\sphinxhyphen{}rendered HTML version of the book will be built in \sphinxcode{\sphinxupquote{doc\sphinxhyphen{}source/\_build/latex/}}.

\end{itemize}

\end{enumerate}

\sphinxAtStartPar
See \sphinxurl{https://jupyterbook.org/basics/building/index.html} for more information.

\begin{DUlineblock}{0em}
\item[] \sphinxstylestrong{\large Credits}
\end{DUlineblock}

\sphinxAtStartPar
This project is created using the open source \sphinxhref{https://jupyterbook.org/}{Jupyter Book project} and the \sphinxhref{https://github.com/executablebooks/cookiecutter-jupyter-book}{executablebooks/cookiecutter\sphinxhyphen{}jupyter\sphinxhyphen{}book template}.

\sphinxAtStartPar
To add: build env \& main software packages (see automation for this…)

\sphinxAtStartPar
\sphinxincludegraphics{{logo}.png}

\sphinxstepscope


\part{Frontmatter}

\sphinxstepscope


\chapter{Overview}
\label{\detokenize{frontmatter/overview_270122:overview}}\label{\detokenize{frontmatter/overview_270122::doc}}

\section{General overview}
\label{\detokenize{frontmatter/overview_270122:general-overview}}
\sphinxAtStartPar
Vol. 3. will focus on analysis techniques for quantum metrology with photoelectrons, including:
\begin{itemize}
\item {} 
\sphinxAtStartPar
Interpreting experimental data.

\item {} 
\sphinxAtStartPar
Extraction/reconstruction/determination of quantum mechanical properties (matrix elements, wavefunctions, density matrices) from experimental data.

\item {} 
\sphinxAtStartPar
Comparison of experimental and theoretical data.

\item {} 
\sphinxAtStartPar
New analysis methodologies \& techniques.

\item {} 
\sphinxAtStartPar
Introduction to newly\sphinxhyphen{}developed software platform (see below).

\end{itemize}


\section{Provisional contents}
\label{\detokenize{frontmatter/overview_270122:provisional-contents}}

\subsection{Part 1: theory \& software}
\label{\detokenize{frontmatter/overview_270122:part-1-theory-software}}
\sphinxAtStartPar
General review \& update of the topic, including recent theory developments.
\begin{enumerate}
\sphinxsetlistlabels{\arabic}{enumi}{enumii}{}{.}%
\item {} 
\sphinxAtStartPar
Introduction

\sphinxAtStartPar
a.  Topic overview.

\sphinxAtStartPar
b.  Context of vol. 3 (following vols. 1 \& 2).

\sphinxAtStartPar
c.  Aims: Vol. 3 in the series will continue the exploration of quantum metrology with photoelectrons, with a focus on numerical analysis techniques, forging a closer link between experimental and theoretical results, and making the methodologies discussed directly accessible via a new software platform/ecosystem.

\item {} 
\sphinxAtStartPar
Quantum metrology software platform/ecosystem overview

\sphinxAtStartPar
a.  Introduction to python packages for simulation, data analysis, and open\sphinxhyphen{}data.

\sphinxAtStartPar
b.  Photoelectron metrology toolkit (PEMtk) package/platform for experimental data processing \& analysis. (See \sphinxhref{https://pemtk.readthedocs.io}{\sphinxstyleemphasis{pemtk.readthedocs.io}}.)

\sphinxAtStartPar
c.  ePSproc package for theory \& simulation. (See \sphinxhref{https://epsproc.readthedocs.io}{\sphinxstyleemphasis{epsproc.readthedocs.io}}.)

\sphinxAtStartPar
d.  ePSdata platform for data/results library (see \sphinxhref{https://phockett.github.io/ePSdata/about.html\#Motivation}{\sphinxstyleemphasis{ePSdata motivations}}).

\item {} 
\sphinxAtStartPar
General method development: geometric tensor treatment of
photoionization, fitting \& matrix\sphinxhyphen{}inversion techniques

\sphinxAtStartPar
a.  Theory development overview \sphinxhyphen{} tensor methods (e.g. \sphinxhref{https:///epsproc.readthedocs.io/en/latest/methodsgeometric\_method\_dev\_pt3\_AFBLM\_090620\_010920\_dev\_bk100920.html}{\sphinxstyleemphasis{ePSproc tensor methods}})

\sphinxAtStartPar
b.  Direct molecular frame reconstruction via matrix\sphinxhyphen{}inversion methods (see Gregory, Margaret, Paul Hockett, Albert Stolow, and Varun Makhija. “Towards Molecular Frame Photoelectron Angular Distributions in Polyatomic Molecules from Lab Frame Coherent Rotational Wavepacket Evolution.” \sphinxstyleemphasis{Journal of Physics B: Atomic, Molecular and Optical Physics} 54, no. 14 (July 2021): 145601.\sphinxhref{https://doi.org/10.1088/1361-6455/ac135f}{\sphinxstyleemphasis{DOI: 10.1088/1361\sphinxhyphen{}6455/ac135f}}.)

\item {} 
\sphinxAtStartPar
Numerical implementation \& analysis platform tools

\sphinxAtStartPar
a.  Tensor methods implementation in ePSproc/PEMtk.

\sphinxAtStartPar
b.  Information content analysis (inc. basis\sphinxhyphen{}set exploration, e.g. \sphinxhref{https://pemtk.readthedocs.io/en/latest/fitting/PEMtk\_fitting\_basis-set\_demo\_050621\_full.html}{\sphinxstyleemphasis{PEMtk fitting demo}}), see also vol. 2, sect. 12.1.

\sphinxAtStartPar
c.  Density matrix analysis. (e.g. \sphinxhref{https://epsproc.readthedocs.io/en/dev/methods/density\_mat\_notes\_demo\_300821.html}{\sphinxstyleemphasis{ePSproc density matrix method dev notes}})

\sphinxAtStartPar
d.  Generalised bootstrapping implementation in PEMtk (see vol. 2, sects. 11.3 \& 12.3)

\end{enumerate}


\subsection{Part 2: numerical examples}
\label{\detokenize{frontmatter/overview_270122:part-2-numerical-examples}}
\sphinxAtStartPar
Open\sphinxhyphen{}source worked examples using the new software platform.
\begin{enumerate}
\sphinxsetlistlabels{\arabic}{enumi}{enumii}{}{.}%
\item {} 
\sphinxAtStartPar
Quantum metrology example: generalised bootstrapping for a homonuclear diatomic scattering system (N2)*

\sphinxAtStartPar
a.  Experimental data overview \& simulation.

\sphinxAtStartPar
b.  Matrix element extraction (bootstrap protocol, see vol. 2, sects. 11.3 \& 12.3) \& statistical analysis.

\sphinxAtStartPar
c.  Direct molecular frame reconstruction via matrix\sphinxhyphen{}inversion methods.

\sphinxAtStartPar
d.  Comparison of methods.

\sphinxAtStartPar
e.  Information content/quantum information analysis. (See vol. 2, sect. 12.1.)

\item {} 
\sphinxAtStartPar
Quantum metrology example: generalised bootstrapping for a heteronuclear scattering system (CO)*

\sphinxAtStartPar
a.  Experimental data overview \& simulation.

\sphinxAtStartPar
b.  Matrix element extraction (bootstrap protocol, see vol. 2, sects. 11.3 \& 12.3) \& statistical analysis.

\sphinxAtStartPar
c.  Direct molecular frame reconstruction via matrix\sphinxhyphen{}inversion methods.

\sphinxAtStartPar
d.  Comparison of methods.

\sphinxAtStartPar
e.  Information content/quantum information analysis. (See vol. 2,
sect. 12.1.)

\item {} 
\sphinxAtStartPar
Quantum metrology example: generalised bootstrapping and
matrix\sphinxhyphen{}inversion methods for a complex/general asymmetric top
scattering system (C2H4 (ethylene))*

\sphinxAtStartPar
a.  Experimental data overview \& simulation.

\sphinxAtStartPar
b.  Matrix element extraction (bootstrap protocol, see vol. 2,
sects. 11.3 \& 12.3) \& statistical analysis.

\sphinxAtStartPar
c.  Direct molecular frame reconstruction via matrix\sphinxhyphen{}inversion
methods.

\sphinxAtStartPar
d.  Comparison of methods.

\sphinxAtStartPar
e.  Information content/quantum information analysis.

\item {} 
\sphinxAtStartPar
Future directions \& outlook

\item {} 
\sphinxAtStartPar
Summary \& conclusions

\end{enumerate}

\sphinxAtStartPar
* Exact choice of “simple” and “complex” systems may change, but should
include a homonuclear diatomic and/or heteronuclear diatomic, and
symmetric and asymmetric top polyatomic systems. May also include an
atomic example.

\sphinxstepscope


\part{Theory \& software}

\sphinxstepscope


\chapter{Introduction}
\label{\detokenize{part1/main_intro_051122:introduction}}\label{\detokenize{part1/main_intro_051122::doc}}
\sphinxAtStartPar
The overall aim of Vol. 3 is to expand, explore, and illustrate, quantum metrology with photoelectrons: specifically, the application of new python\sphinxhyphen{}based tools to tackle problems in matrix element retrieval. The book itself is written as a set of Jupyter Notebooks, hence all the material herein is available directly to readers, and can be run locally to further explore the topic, or adapt the methodology to new problems

\sphinxAtStartPar
Whilst this volume aims to provide a self\sphinxhyphen{}contained text, and computational examples which may be used without extensive background knowledge, a brief introduction to the core physics and some recent extensions is presented herein. The unfamiliar reader is referred to Volume 1 of the series for a more detailed presentation, and gateway to the literature {[}\hyperlink{cite.backmatter/bibliography:id52}{1}{]}. Following the topical introduction, the remainder of Part I introduces the main computational and software tools, recent theory developments, and concludes with a general overview for approaching matrix element retrieval numerically.

\sphinxAtStartPar
Part II details the application of these tools to a few specific cases, starting with a (relatively) simple homonuclear diatomic example, then escalating to a polyatomic asymetric top case.


\section{Topical introduction}
\label{\detokenize{part1/main_intro_051122:topical-introduction}}

\section{Context \& aims for Vol. 3}
\label{\detokenize{part1/main_intro_051122:context-aims-for-vol-3}}
\sphinxAtStartPar
As noted previously, Vol. 3 is somewhat distinct from the previous volumes in the series; although involving computational elements, Vols. 1 \& 2 {[}\hyperlink{cite.backmatter/bibliography:id52}{1}, \hyperlink{cite.backmatter/bibliography:id53}{2}{]} are more traditional publications. The material presented in this volume aims to continue the exploration of quantum metrology with photoelectrons, with a focus on numerical analysis techniques, forging a closer link between experimental and theoretical results, and making the methodologies discussed directly accessible via a new software platform/ecosystem. In order to fulfil this aim, Vol. 3 is a computational/computable document, with code directly available to readers. Each chapter or section is composed of a Jupyter Notebook (\sphinxcode{\sphinxupquote{.ipynb}}), each of which can be modified and used independently.

\sphinxAtStartPar
To facilitate code transparency and reuse, the book is available via a Github repository, \sphinxhref{https://github.com/phockett/Quantum-Metrology-with-Photoelectrons-Vol3}{Quantum Metrology Vol. 3}. An HTML version is also available, which includes interactive figures. A full introduction to the relevant tool\sphinxhyphen{}chain, including installation instructions, can be found in \hyperref[\detokenize{part1/platform_intro_071122:chpt-platformintro}]{Chapter \ref{\detokenize{part1/platform_intro_071122:chpt-platformintro}}:} {\hyperref[\detokenize{part1/platform_intro_071122:chpt-platformintro}]{\sphinxcrossref{\DUrole{std,std-ref}{Quantum metrology software platform/ecosystem overview}}}}.

\sphinxstepscope


\chapter{Quantum metrology software platform/ecosystem overview}
\label{\detokenize{part1/platform_intro_071122:quantum-metrology-software-platform-ecosystem-overview}}\label{\detokenize{part1/platform_intro_071122:chpt-platformintro}}\label{\detokenize{part1/platform_intro_071122::doc}}
\sphinxAtStartPar
STUB

\sphinxAtStartPar
In recent years, a unified Python codebase/ecosystem/platform has been in development to tackle various aspects of photoionization problems, including \sphinxstyleemphasis{ab initio} computations and experimental data handling, and (generalised) matrix element retrieval methods. The eponymous \sphinxstyleemphasis{Quantum Metrology with Photoelectrons} platform is introduced here, and is used for the analysis herein. The main aim of the platform is to provide a unifying data platform, and analysis routines, for photoelectron metrology, including new methods and tools, as well as a unifying bridge between these and existing tools. \hyperref[\detokenize{part1/platform_intro_071122:qm-platform-diag}]{Fig.\@ \ref{\detokenize{part1/platform_intro_071122:qm-platform-diag}}} provides a general overview of some of the main tools and tasks/layers.

\sphinxAtStartPar
As of late 2022, the new parts of the platform \sphinxhyphen{} primarily the \sphinxhref{https://github.com/phockett/PEMtk}{Photoelectron Metrology Toolkit} {[}\hyperlink{cite.backmatter/bibliography:id57}{3}{]} library \sphinxhyphen{} implement general data handling (although not a full experimental analysis toolchain), matrix element handling and retrieval, which will be the main topic of this volume.
In the future, it is hoped that the platform will be extended to other theoretical and experimental methods, including full experimental data handling.


\section{Analysis components}
\label{\detokenize{part1/platform_intro_071122:analysis-components}}\label{\detokenize{part1/platform_intro_071122:sect-platform-analysis}}
\sphinxAtStartPar
The two main components of the platform for analysis tasks, as used herein, are:
\begin{itemize}
\item {} 
\sphinxAtStartPar
The \sphinxhref{https://github.com/phockett/PEMtk}{Photoelectron Metrology Toolkit} {[}\hyperlink{cite.backmatter/bibliography:id57}{3}{]} (PEMtk) codebase aims to provide various general data handling routines for photoionization problems. At the time of writing, simulation of observables and fitting routines are implemented, along with some basic utility functions.
Much of this is detailed herein, and more technical details and ongoing documentation case be found in the \sphinxhref{https://pemtk.readthedocs.io}{PEMtk documentation} {[}\hyperlink{cite.backmatter/bibliography:id56}{4}{]}.

\item {} 
\sphinxAtStartPar
The \sphinxhref{https://epsproc.readthedocs.io}{ePSproc codebase} {[}\hyperlink{cite.backmatter/bibliography:id45}{5}, \hyperlink{cite.backmatter/bibliography:id41}{6}, \hyperlink{cite.backmatter/bibliography:id40}{7}{]} aims to provide methods for post\sphinxhyphen{}processing with \sphinxstyleemphasis{ab initio} radial dipole matrix
elements from \sphinxhref{https://epolyscat.droppages.com/}{ePolyScat (ePS)} {[}\hyperlink{cite.backmatter/bibliography:id63}{8}, \hyperlink{cite.backmatter/bibliography:id43}{9}, \hyperlink{cite.backmatter/bibliography:id66}{10}, \hyperlink{cite.backmatter/bibliography:id64}{11}{]}, or equivalent matrix elements from other sources (dedicated support for R\sphinxhyphen{}matrix results from \sphinxhref{https://gitlab.com/Uk-amor/RMT/rmt}{the RMT suite} {[}\hyperlink{cite.backmatter/bibliography:id39}{12}, \hyperlink{cite.backmatter/bibliography:id68}{13}{]} is in development).
The core functionality includes the computation of AF and MF observables. Manual computation without known matrix elements is also possible, e.g. for investigating
limiting cases, or data analysis and fitting \sphinxhyphen{} hence these routines also provide the backend functionality for PEMtk fitting routines. Again more technical details can be found in the \sphinxhref{https://epsproc.readthedocs.io}{ePSproc documentation} {[}\hyperlink{cite.backmatter/bibliography:id40}{7}{]}.

\end{itemize}

\begin{figure}[htbp]
\centering
\capstart

\noindent\sphinxincludegraphics{{QM_unified_schema_wrapped_280820_gv}.png}
\caption{Quantum metrology with photoelectrons ecosystem overview.}\label{\detokenize{part1/platform_intro_071122:qm-platform-diag}}\end{figure}


\section{Additional tools}
\label{\detokenize{part1/platform_intro_071122:additional-tools}}\label{\detokenize{part1/platform_intro_071122:sect-platform-othertools}}
\sphinxAtStartPar
Other tools listed in \hyperref[\detokenize{part1/platform_intro_071122:qm-platform-diag}]{Fig.\@ \ref{\detokenize{part1/platform_intro_071122:qm-platform-diag}}} include:
\begin{itemize}
\item {} 
\sphinxAtStartPar
Quantum chemistry layer. The starting point for \sphinxstyleemphasis{ab initio} computations. For the examples herein, all computations made use of \sphinxhref{http://www.msg.ameslab.gov/gamess/}{Gamess (“The General Atomic and Molecular Electronic Structure System”)} {[}\hyperlink{cite.backmatter/bibliography:id42}{14}, \hyperlink{cite.backmatter/bibliography:id44}{15}{]} for electronic structure computations, and inputs to ePolyScat.

\item {} 
\sphinxAtStartPar
\sphinxhref{https://epolyscat.droppages.com/}{ePolyScat (ePS)} {[}\hyperlink{cite.backmatter/bibliography:id63}{8}, \hyperlink{cite.backmatter/bibliography:id43}{9}, \hyperlink{cite.backmatter/bibliography:id66}{10}, \hyperlink{cite.backmatter/bibliography:id64}{11}{]} is an open\sphinxhyphen{}source tool for numerical computation of electron\sphinxhyphen{}molecule scattering \& photoionization by Lucchese \& coworkers. All matrix elements used herein were obtained via ePS calculations. For more details see \sphinxhref{https://epolyscat.droppages.com/}{ePolyScat website and manual} {[}\hyperlink{cite.backmatter/bibliography:id64}{11}{]} and Refs. {[}\hyperlink{cite.backmatter/bibliography:id63}{8}, \hyperlink{cite.backmatter/bibliography:id43}{9}, \hyperlink{cite.backmatter/bibliography:id66}{10}{]}.

\item {} 
\sphinxAtStartPar
\sphinxhref{https://phockett.github.io/ePSdata/about.html}{ePSdata} {[}\hyperlink{cite.backmatter/bibliography:id54}{16}{]} is an open\sphinxhyphen{}data/open\sphinxhyphen{}science collection of ePS + ePSproc results.
\begin{itemize}
\item {} 
\sphinxAtStartPar
ePSdata collects ePS datasets, post\sphinxhyphen{}processed via ePSproc (Python) in \sphinxhref{https://jupyter.org}{Jupyter notebooks}, for a full open\sphinxhyphen{}data/open\sphinxhyphen{}science transparent pipeline.

\end{itemize}
\begin{itemize}
\item {} 
\sphinxAtStartPar
Source notebooks are available on the \sphinxhref{https://github.com/phockett/ePSdata/}{ePSdata} {[}\hyperlink{cite.backmatter/bibliography:id54}{16}{]} \sphinxhref{https://github.com/phockett/ePSdata/}{Github project repository}, and notebooks + datasets via \sphinxhref{https://zenodo.org/search?page=1\&size=20\&q=hockett\&keywords=Data}{ePSdata Zenodo} {[}\hyperlink{cite.backmatter/bibliography:id55}{17}{]}. Each notebook + dataset is given a Zenodo DOI for full traceability, and notebooks are versioned on Github.

\item {} 
\sphinxAtStartPar
Note: ePSdata may also be linked or mirrored on the existing \sphinxhref{https://osf.io/psjxt/}{ePolyScat Collected Results OSF project}, but will effectively supercede those pages.

\item {} 
\sphinxAtStartPar
All results are released under \sphinxhref{https://creativecommons.org/licenses/by-nc-sa/4.0/}{Creative Commons Attribution\sphinxhyphen{}NonCommercial\sphinxhyphen{}ShareAlike 4.0 (CC BY\sphinxhyphen{}NC\sphinxhyphen{}SA 4.0) license}, and are part of an ongoing \sphinxhref{http://femtolab.ca/?p=877}{Open Science initiative}.

\end{itemize}

\end{itemize}


\section{Docker deployments}
\label{\detokenize{part1/platform_intro_071122:docker-deployments}}\label{\detokenize{part1/platform_intro_071122:sect-platform-docker}}
\sphinxAtStartPar
A Docker\sphinxhyphen{}based distribution of various codes for tackling
photoionization problems is also available from the \sphinxhref{https://github.com/phockett/open-photoionization-docker-stacks}{Open Photoionization Docker Stacks} {[}\hyperlink{cite.backmatter/bibliography:id59}{18}{]}
project, which aims to make a range of these tools more accessible to
interested researchers, and fully cross\sphinxhyphen{}platform/portable. The project currently includes Docker builds for \sphinxcode{\sphinxupquote{ePS}}, \sphinxcode{\sphinxupquote{ePSproc}} and \sphinxcode{\sphinxupquote{PEMtk}}.


\section{General discussion}
\label{\detokenize{part1/platform_intro_071122:general-discussion}}\label{\detokenize{part1/platform_intro_071122:sect-platform-general}}
\sphinxAtStartPar
Note that, at the time of writing, rotational wavepacket simulation is
not yet implemented in the PEMtk suite, and these must be obtained via
other codes. An intial build of the \sphinxcode{\sphinxupquote{limapack}} suite for rotational wavepacket simulations is currently part of the \sphinxhref{https://github.com/phockett/open-photoionization-docker-stacks}{Open Photoionization Docker Stacks} {[}\hyperlink{cite.backmatter/bibliography:id59}{18}{]}, but has yet to be tested.

\sphinxstepscope


\chapter{Theory}
\label{\detokenize{part1/theory_101122:theory}}\label{\detokenize{part1/theory_101122:chpt-theory}}\label{\detokenize{part1/theory_101122::doc}}
\sphinxAtStartPar
STUB

\sphinxAtStartPar
TODO: plotly HV wrappers, see \sphinxurl{https://github.com/executablebooks/jupyter-book/issues/1815}

\sphinxAtStartPar
UPDATE: now in place, working in Render\sphinxhyphen{}debug notebook, but not here \sphinxhyphen{} even after clean build? Weird.

\sphinxAtStartPar
TODO: Fix plots (HV \& plotly) in PDF output, currently missing here and Render\sphinxhyphen{}debug notebook.

\sphinxAtStartPar
17/11/22 Rendering now OK for Plotly, ongoing notes: \sphinxurl{https://github.com/phockett/Quantum-Metrology-with-Photoelectrons-Vol3/issues/2}


\section{Observables: photoelectron flux in the LF and MF}
\label{\detokenize{part1/theory_101122:observables-photoelectron-flux-in-the-lf-and-mf}}\label{\detokenize{part1/theory_101122:sect-theory-observables}}
\sphinxAtStartPar
The observables of interest \sphinxhyphen{} the photoelectron flux as a function of energy, ejection angle, and time \sphinxhyphen{} can be written quite generally as an expansion in spherical harmonics:
\begin{equation}\label{equation:part1/theory_101122:eq:AF-PAD-general}
\begin{split}
\begin{align}
\bar{I}(\epsilon,t,\theta,\phi)=\sum_{L=0}^{2n}\sum_{M=-L}^{L}\bar{\beta}_{L,M}(\epsilon,t)Y_{L,M}(\theta,\phi)
\end{align}
\end{split}
\end{equation}
\sphinxAtStartPar
Here the flux in the laboratory frame (LF) or aligned frame (AF) is denoted \(\bar{I}(\epsilon,t,\theta,\phi)\), with the bar signifying ensemble averaging, and the molecular frame flux by \(I(\epsilon,t,\theta,\phi)\). Similarly, the expansion parameters \(\bar{\beta}_{L,M}(\epsilon,t)\) include a bar for the LF/AF case. These observables are generally termed photoelectron angular distributions (PADs), often with a prefix denoting the reference frame, e.g. LFPADs, MFPADs, and the associated expansion parameters \(\bar{\beta}_{L,M}(\epsilon,t)\) are generically termed “anisotropy” parameters. The polar coordinate system \((\theta,\phi)\) is referenced to
an experimentally\sphinxhyphen{}defined axis in the LF/AF case (usually defined by the laser polarization), and the molecular symmetry axis in the MF. Some arbitrary examples are given in \hyperref[\detokenize{part1/theory_101122:fig-pads-example}]{Fig.\@ \ref{\detokenize{part1/theory_101122:fig-pads-example}}}, which illustrates both a range of distributions of increasing complexity, and some basic code to set \(\beta_{L,M}\) parameters and visualise them; the values used as tabulated in  \hyperref[\detokenize{part1/theory_101122:blm-tab}]{Fig.\@ \ref{\detokenize{part1/theory_101122:blm-tab}}}.

\begin{sphinxuseclass}{cell}
\begin{sphinxuseclass}{tag_hide-output}\begin{sphinxVerbatimInput}

\begin{sphinxuseclass}{cell_input}
\begin{sphinxVerbatim}[commandchars=\\\{\}]
\PYG{c+c1}{\PYGZsh{} Plot some distributions from specified BLMs}

\PYG{c+c1}{\PYGZsh{} Set specific LM coeffs by list with setBLMs, items are [l,m,value]}
\PYG{k+kn}{from} \PYG{n+nn}{epsproc}\PYG{n+nn}{.}\PYG{n+nn}{sphCalc} \PYG{k+kn}{import} \PYG{n}{setBLMs}

\PYG{c+c1}{\PYGZsh{} BLM = setBLMs([[0,0,1],[1,1,1],[2,2,1]])}
\PYG{c+c1}{\PYGZsh{} BLM = setBLMs([[0,0,1,1,1],[1,1,1,0.5,0.2],[2,2,1,1,0.2]])   \PYGZsh{} Note different index}
\PYG{n}{BLM} \PYG{o}{=} \PYG{n}{setBLMs}\PYG{p}{(}\PYG{p}{[}\PYG{p}{[}\PYG{l+m+mi}{0}\PYG{p}{,}\PYG{l+m+mi}{0}\PYG{p}{,}\PYG{l+m+mi}{1}\PYG{p}{,}\PYG{l+m+mi}{1}\PYG{p}{,}\PYG{l+m+mi}{1}\PYG{p}{,}\PYG{l+m+mi}{1}\PYG{p}{]}\PYG{p}{,}\PYG{p}{[}\PYG{l+m+mi}{1}\PYG{p}{,}\PYG{l+m+mi}{1}\PYG{p}{,}\PYG{l+m+mi}{0}\PYG{p}{,}\PYG{l+m+mf}{0.5}\PYG{p}{,}\PYG{l+m+mf}{0.8}\PYG{p}{,}\PYG{l+m+mi}{1}\PYG{p}{]}\PYG{p}{,}\PYG{p}{[}\PYG{l+m+mi}{2}\PYG{p}{,}\PYG{l+m+mi}{0}\PYG{p}{,}\PYG{l+m+mi}{1}\PYG{p}{,}\PYG{l+m+mf}{0.5}\PYG{p}{,}\PYG{l+m+mi}{0}\PYG{p}{,}\PYG{l+m+mi}{0}\PYG{p}{]}\PYG{p}{,}
               \PYG{p}{[}\PYG{l+m+mi}{4}\PYG{p}{,}\PYG{l+m+mi}{2}\PYG{p}{,}\PYG{l+m+mi}{0}\PYG{p}{,}\PYG{l+m+mi}{0}\PYG{p}{,}\PYG{l+m+mi}{0}\PYG{p}{,}\PYG{l+m+mf}{0.5}\PYG{p}{]}\PYG{p}{,}\PYG{p}{[}\PYG{l+m+mi}{4}\PYG{p}{,}\PYG{o}{\PYGZhy{}}\PYG{l+m+mi}{2}\PYG{p}{,}\PYG{l+m+mi}{0}\PYG{p}{,}\PYG{l+m+mi}{0}\PYG{p}{,}\PYG{l+m+mi}{0}\PYG{p}{,}\PYG{l+m+mf}{0.5}\PYG{p}{]}\PYG{p}{]}\PYG{p}{)}

\PYG{c+c1}{\PYGZsh{} Set the backend to \PYGZsq{}pl\PYGZsq{} for an interactive surface plot with Plotly}
\PYG{c+c1}{\PYGZsh{} NOTE PL FIG RETURN BROKEN FOR THIS CASE (ePSproc v1.3.1), so run sphSumPlotX too.}
\PYG{n}{dataPlot}\PYG{p}{,} \PYG{n}{figObj} \PYG{o}{=} \PYG{n}{ep}\PYG{o}{.}\PYG{n}{sphFromBLMPlot}\PYG{p}{(}\PYG{n}{BLM}\PYG{p}{,} \PYG{n}{facetDim}\PYG{o}{=}\PYG{l+s+s1}{\PYGZsq{}}\PYG{l+s+s1}{t}\PYG{l+s+s1}{\PYGZsq{}}\PYG{p}{,} \PYG{n}{plotFlag} \PYG{o}{=} \PYG{k+kc}{False}\PYG{p}{,} \PYG{n}{backend} \PYG{o}{=} \PYG{n}{plotBackend}\PYG{p}{)}\PYG{p}{;}
\PYG{n}{figObj} \PYG{o}{=} \PYG{n}{ep}\PYG{o}{.}\PYG{n}{sphSumPlotX}\PYG{p}{(}\PYG{n}{dataPlot}\PYG{p}{,}\PYG{n}{facetDim}\PYG{o}{=}\PYG{l+s+s1}{\PYGZsq{}}\PYG{l+s+s1}{t}\PYG{l+s+s1}{\PYGZsq{}}\PYG{p}{,} \PYG{n}{plotFlag} \PYG{o}{=} \PYG{k+kc}{False}\PYG{p}{,} \PYG{n}{backend} \PYG{o}{=} \PYG{n}{plotBackend}\PYG{p}{)}\PYG{p}{;}

\PYG{c+c1}{\PYGZsh{} And GLUE for display later with caption}
\PYG{c+c1}{\PYGZsh{} from myst\PYGZus{}nb import glue}
\PYG{c+c1}{\PYGZsh{} glue(\PYGZdq{}padExamplePlot\PYGZdq{}, figObj[0], display=False);}
\PYG{c+c1}{\PYGZsh{} Glue with Plotly wrapper.}
\PYG{c+c1}{\PYGZsh{} gluePlotly(\PYGZdq{}padExamplePlot\PYGZdq{}, figObj[0])   \PYGZsh{} Working in Render test notebook, but not here? Issue with subplots?}

\PYG{c+c1}{\PYGZsh{} Test in separate cell...}
\PYG{n}{gluePlotly}\PYG{p}{(}\PYG{l+s+s2}{\PYGZdq{}}\PYG{l+s+s2}{padExamplePlot}\PYG{l+s+s2}{\PYGZdq{}}\PYG{p}{,} \PYG{n}{figObj}\PYG{p}{[}\PYG{l+m+mi}{0}\PYG{p}{]}\PYG{p}{)}   \PYG{c+c1}{\PYGZsh{} Working in Render test notebook, but not here? Issue with subplots?}
\end{sphinxVerbatim}

\end{sphinxuseclass}\end{sphinxVerbatimInput}

\end{sphinxuseclass}
\end{sphinxuseclass}
\begin{figure}[htbp]
\centering
\capstart
\caption{Examples of angular distributions (expansions in spherical harmonics \(Y_{L,M}\)), for a range of cases. Note that up\sphinxhyphen{}down asymmetry is associated with odd\sphinxhyphen{}\(l\) contributions, and breaking of cylindrical symmetry with \(m\neq0\) terms.}\label{\detokenize{part1/theory_101122:fig-pads-example}}\end{figure}

\begin{figure}[htbp]
\centering
\capstart
\caption{Values used for the plots in \hyperref[\detokenize{part1/theory_101122:fig-pads-example}]{Fig.\@ \ref{\detokenize{part1/theory_101122:fig-pads-example}}}}\label{\detokenize{part1/theory_101122:blm-tab}}\end{figure}

\sphinxAtStartPar
In general, the spherical harmonic rank and order \((L,M)\) of Eq. \eqref{equation:part1/theory_101122:eq:AF-PAD-general} are constrained by experimental factors in the LF/AF, and \(n\) is effectively limited by the molecular alignment (which is correlated with the photon\sphinxhyphen{}order for gas phase experiments, or conservation of angular momentum in the LF more generally {[}{]}), but in the MF is defined by the maximum continuum angular momentum \(n=l_{max}\) imparted by the scattering event {[}{]}.

\sphinxAtStartPar
For basic cases these limits may be low: for instance, a simple 1\sphinxhyphen{}photon photoionization event (\(n=1\)) from an isotropic ensemble (zero net ensemble angular momentum) defines \(L_{max}=2\); for cylindrically symmetric cases (i.e. \(D_{\infty h}\) symmetry) \(M=0\) only. For MF cases, \(l_{max}=4\) is often given as a reasonable rule\sphinxhyphen{}of\sphinxhyphen{}thumb for the continuum \sphinxhyphen{} hence \(L_{max}=8\) \sphinxhyphen{} although in practice higher\sphinxhyphen{}\(l\) may be populated. Some realistic example cases are discussed later (\sphinxstylestrong{PART II}), see also ref. {[}\hyperlink{cite.backmatter/bibliography:id52}{1}{]} for more discussion and complex examples.

\sphinxAtStartPar
In general, these observables may also be dependent on various other parameters; in Eq. \eqref{equation:part1/theory_101122:eq:AF-PAD-general} two such parameters, \((\epsilon,t)\), are included, as the usual variables of interest. Usually \(\epsilon\) denotes the photoelectron energy, and \(t\) is used in the case of time\sphinxhyphen{}dependent (usually pump\sphinxhyphen{}probe) measurements. As discussed below (\hyperref[\detokenize{part1/theory_101122:sec-dynamics-intro}]{Sect.\@ \ref{\detokenize{part1/theory_101122:sec-dynamics-intro}}}), the origin of such dependencies may be complicated but, in general, the associated photoionization matrix elements are energy\sphinxhyphen{}dependent, and time\sphinxhyphen{}dependence may also appear for a number of intrinsic or extrinsic (experimental) reasons, e.g. electronic or nuclear dynamics, rotational (alignment) dynamics, electric field dynamics etc. In many cases only one particular aspect may be of interest, so \(t\) can be used as a generic label to index changes as per \hyperref[\detokenize{part1/theory_101122:fig-pads-example}]{Fig.\@ \ref{\detokenize{part1/theory_101122:fig-pads-example}}}.


\subsection{Symmetrized harmonics}
\label{\detokenize{part1/theory_101122:symmetrized-harmonics}}\label{\detokenize{part1/theory_101122:sec-theory-sym-harm-into}}
\sphinxAtStartPar
Symmetrized (or generalised) harmonics, which essentially provide correctly symmetrized expansions of spherical harmonics (\(Y_{lm}\)) functions for a given irreducible representation, \(\Gamma\), can be defined by linear combinations of spherical harmonics (Refs. {[}{]} as below):
\label{equation:part1/theory_101122:0c4f30fd-5788-462f-969f-2965228e2865}\begin{equation}
X_{hl}^{\Gamma\mu*}(\theta,\phi)=\sum_{\lambda}b_{hl\lambda}^{\Gamma\mu}Y_{l,\lambda}(\theta,\phi)\label{eq:symm-harmonics}
\end{equation}
\sphinxAtStartPar
where:
\begin{itemize}
\item {} 
\sphinxAtStartPar
\(\Gamma\) is an irreducible representation,

\item {} 
\sphinxAtStartPar
(\(l\), \(\lambda\)) define the usual spherical harmonic indicies (rank, order)

\item {} 
\sphinxAtStartPar
\(b_{hl\lambda}^{\Gamma\mu}\) are symmetrization coefficients,

\item {} 
\sphinxAtStartPar
index \(\mu\) allows for indexing of degenerate components,

\item {} 
\sphinxAtStartPar
\(h\) indexs cases where multiple components are required with all other quantum numbers identical.

\end{itemize}

\sphinxAtStartPar
The exact form of these coefficients will depend on the point\sphinxhyphen{}group of the system, see, e.g. Refs. {[}{]}. Numerical routines for the generation of symmetrized harmonics are implemented in PEMtk: point\sphinxhyphen{}groups, character table generation and symmetrization (computing \(b_{hl\lambda}^{\Gamma\mu}\) parameters) is handled by \sphinxhref{https://github.com/mcodev31/libmsym}{libmsym} {[}\hyperlink{cite.backmatter/bibliography:id62}{19}{]}; additional handling also makes use of \sphinxhref{https://shtools.oca.eu}{pySHtools} {[}\hyperlink{cite.backmatter/bibliography:id70}{20}, \hyperlink{cite.backmatter/bibliography:id71}{21}{]}. A brief example is given below, see the \sphinxhref{https://pemtk.readthedocs.io}{PEMtk documentation} {[}\hyperlink{cite.backmatter/bibliography:id56}{4}{]} for more details.

\begin{sphinxuseclass}{cell}
\begin{sphinxuseclass}{tag_hide-output}\begin{sphinxVerbatimInput}

\begin{sphinxuseclass}{cell_input}
\begin{sphinxVerbatim}[commandchars=\\\{\}]
\PYG{c+c1}{\PYGZsh{} Import class}
\PYG{k+kn}{from} \PYG{n+nn}{pemtk}\PYG{n+nn}{.}\PYG{n+nn}{sym}\PYG{n+nn}{.}\PYG{n+nn}{symHarm} \PYG{k+kn}{import} \PYG{n}{symHarm}

\PYG{c+c1}{\PYGZsh{} Compute hamronics for Td, lmax=4}
\PYG{n}{symObj} \PYG{o}{=} \PYG{n}{symHarm}\PYG{p}{(}\PYG{l+s+s1}{\PYGZsq{}}\PYG{l+s+s1}{Td}\PYG{l+s+s1}{\PYGZsq{}}\PYG{p}{,}\PYG{l+m+mi}{4}\PYG{p}{)}

\PYG{c+c1}{\PYGZsh{} Character tables can be displayed}
\PYG{n}{symObj}\PYG{o}{.}\PYG{n}{printCharacterTable}\PYG{p}{(}\PYG{p}{)}
\end{sphinxVerbatim}

\end{sphinxuseclass}\end{sphinxVerbatimInput}

\end{sphinxuseclass}
\end{sphinxuseclass}
\begin{sphinxuseclass}{cell}
\begin{sphinxuseclass}{tag_hide-output}\begin{sphinxVerbatimInput}

\begin{sphinxuseclass}{cell_input}
\begin{sphinxVerbatim}[commandchars=\\\{\}]
\PYG{c+c1}{\PYGZsh{} The full set of expansion parameters can be tabulated}
\PYG{c+c1}{\PYGZsh{} pd.set\PYGZus{}option(\PYGZsq{}display.max\PYGZus{}rows\PYGZsq{}, 1)}
\PYG{n}{symObj}\PYG{o}{.}\PYG{n}{displayXlm}\PYG{p}{(}\PYG{p}{)}  \PYG{c+c1}{\PYGZsh{} Display values (note this defaults to REAL harmonics)}
\PYG{c+c1}{\PYGZsh{} symObj.displayXlm(YlmType=\PYGZsq{}comp\PYGZsq{})   \PYGZsh{} Display values for COMPLEX harmonic expansion.}
\end{sphinxVerbatim}

\end{sphinxuseclass}\end{sphinxVerbatimInput}

\end{sphinxuseclass}
\end{sphinxuseclass}
\begin{sphinxuseclass}{cell}\begin{sphinxVerbatimInput}

\begin{sphinxuseclass}{cell_input}
\begin{sphinxVerbatim}[commandchars=\\\{\}]
\PYG{c+c1}{\PYGZsh{} To plot using ePSproc/PEMtk class, these values can be converted to ePSproc BLM data type...}

\PYG{c+c1}{\PYGZsh{} Run conversion \PYGZhy{} the default is to set the coeffs to the \PYGZsq{}BLM\PYGZsq{} data type}
\PYG{n}{symObj}\PYG{o}{.}\PYG{n}{toePSproc}\PYG{p}{(}\PYG{p}{)}

\PYG{c+c1}{\PYGZsh{} Set to new key in data class}
\PYG{n}{data}\PYG{o}{.}\PYG{n}{data}\PYG{p}{[}\PYG{l+s+s1}{\PYGZsq{}}\PYG{l+s+s1}{symHarm}\PYG{l+s+s1}{\PYGZsq{}}\PYG{p}{]} \PYG{o}{=} \PYG{p}{\PYGZob{}}\PYG{p}{\PYGZcb{}}

\PYG{k}{for} \PYG{n}{dataType} \PYG{o+ow}{in} \PYG{p}{[}\PYG{l+s+s1}{\PYGZsq{}}\PYG{l+s+s1}{BLM}\PYG{l+s+s1}{\PYGZsq{}}\PYG{p}{]}\PYG{p}{:}  \PYG{c+c1}{\PYGZsh{}[\PYGZsq{}matE\PYGZsq{},\PYGZsq{}BLM\PYGZsq{}]:}
    \PYG{n}{data}\PYG{o}{.}\PYG{n}{data}\PYG{p}{[}\PYG{l+s+s1}{\PYGZsq{}}\PYG{l+s+s1}{symHarm}\PYG{l+s+s1}{\PYGZsq{}}\PYG{p}{]}\PYG{p}{[}\PYG{n}{dataType}\PYG{p}{]} \PYG{o}{=} \PYG{n}{symObj}\PYG{o}{.}\PYG{n}{coeffs}\PYG{p}{[}\PYG{n}{dataType}\PYG{p}{]}\PYG{p}{[}\PYG{l+s+s1}{\PYGZsq{}}\PYG{l+s+s1}{b (comp)}\PYG{l+s+s1}{\PYGZsq{}}\PYG{p}{]}  \PYG{c+c1}{\PYGZsh{} Select expansion in complex harmonics}
    \PYG{n}{data}\PYG{o}{.}\PYG{n}{data}\PYG{p}{[}\PYG{l+s+s1}{\PYGZsq{}}\PYG{l+s+s1}{symHarm}\PYG{l+s+s1}{\PYGZsq{}}\PYG{p}{]}\PYG{p}{[}\PYG{n}{dataType}\PYG{p}{]}\PYG{o}{.}\PYG{n}{attrs} \PYG{o}{=} \PYG{n}{symObj}\PYG{o}{.}\PYG{n}{coeffs}\PYG{p}{[}\PYG{n}{dataType}\PYG{p}{]}\PYG{o}{.}\PYG{n}{attrs}
    
\PYG{c+c1}{\PYGZsh{} Plot full harmonics expansions, plots by symmetry}
\PYG{c+c1}{\PYGZsh{} Note \PYGZsq{}squeeze=True\PYGZsq{} to force drop of singleton dims may be required.}
\PYG{n}{data}\PYG{o}{.}\PYG{n}{padPlot}\PYG{p}{(}\PYG{n}{keys}\PYG{o}{=}\PYG{l+s+s1}{\PYGZsq{}}\PYG{l+s+s1}{symHarm}\PYG{l+s+s1}{\PYGZsq{}}\PYG{p}{,}\PYG{n}{dataType}\PYG{o}{=}\PYG{l+s+s1}{\PYGZsq{}}\PYG{l+s+s1}{BLM}\PYG{l+s+s1}{\PYGZsq{}}\PYG{p}{,} \PYG{n}{facetDims} \PYG{o}{=} \PYG{p}{[}\PYG{l+s+s1}{\PYGZsq{}}\PYG{l+s+s1}{Cont}\PYG{l+s+s1}{\PYGZsq{}}\PYG{p}{]}\PYG{p}{,} \PYG{n}{squeeze} \PYG{o}{=} \PYG{k+kc}{True}\PYG{p}{,} \PYG{n}{backend}\PYG{o}{=}\PYG{n}{plotBackend}\PYG{p}{)}
\end{sphinxVerbatim}

\end{sphinxuseclass}\end{sphinxVerbatimInput}

\end{sphinxuseclass}

\section{Photoionization dynamics}
\label{\detokenize{part1/theory_101122:photoionization-dynamics}}\label{\detokenize{part1/theory_101122:sec-dynamics-intro}}
\sphinxAtStartPar
The core physics of photoionization has been covered extensively in the literature, and only a very brief overview is provided here with sufficient detail to introduce the metrology/reconstruction/retrieval problem; the reader is referred to Vol. 1 {[}\hyperlink{cite.backmatter/bibliography:id52}{1}{]} (and refs. therein) for further details and general discussion.

\sphinxAtStartPar
Photoionization can be described by the coupling of an initial state of the system to a particular final state (photoion(s) plus free photoelectron(s)), coupled by an electric field/photon. Very generically, this can be written as a matrix element \(\langle\Psi_i|\hat{\Gamma}(\boldsymbol{\mathbf{E}})|\Psi_f\rangle\), where \(\hat{\Gamma}(\boldsymbol{\mathbf{E}})\) defines the light\sphinxhyphen{}matter coupling operator (depending on the electric field \(\boldsymbol{\mathbf{E}}\)), and \(\Psi_i\), \(\Psi_f\) the total wavefunctions of the initial and final states respectively.

\sphinxstepscope


\part{Backmatter}

\sphinxstepscope


\chapter{Bibliography}
\label{\detokenize{backmatter/bibliography:bibliography}}\label{\detokenize{backmatter/bibliography::doc}}\phantomsection\label{\detokenize{backmatter/bibliography:id1}}
\sphinxstepscope


\part{Test pages}

\begin{sphinxthebibliography}{10}
\bibitem[1]{backmatter/bibliography:id52}
\sphinxAtStartPar
Paul Hockett. \sphinxstyleemphasis{Quantum Metrology with Photoelectrons, Volume 1: Foundations}. IOP Publishing, 2018. ISBN 978\sphinxhyphen{}1\sphinxhyphen{}68174\sphinxhyphen{}684\sphinxhyphen{}5. URL: \sphinxurl{http://iopscience.iop.org/book/978-1-6817-4684-5}, \sphinxhref{https://doi.org/10.1088/978-1-6817-4684-5}{doi:10.1088/978\sphinxhyphen{}1\sphinxhyphen{}6817\sphinxhyphen{}4684\sphinxhyphen{}5}.
\bibitem[2]{backmatter/bibliography:id53}
\sphinxAtStartPar
Paul Hockett. \sphinxstyleemphasis{Quantum Metrology with Photoelectrons, Volume 2: Applications and Advances}. IOP Publishing, 2018. ISBN 978\sphinxhyphen{}1\sphinxhyphen{}68174\sphinxhyphen{}688\sphinxhyphen{}3. URL: \sphinxurl{http://iopscience.iop.org/book/978-1-6817-4688-3}, \sphinxhref{https://doi.org/10.1088/978-1-6817-4688-3}{doi:10.1088/978\sphinxhyphen{}1\sphinxhyphen{}6817\sphinxhyphen{}4688\sphinxhyphen{}3}.
\bibitem[3]{backmatter/bibliography:id57}
\sphinxAtStartPar
Paul Hockett. Photoelectron Metrology Toolkit (PEMtk) Github repository. 2021. URL: \sphinxurl{https://github.com/phockett/PEMtk} (visited on 2022\sphinxhyphen{}02\sphinxhyphen{}18).
\bibitem[4]{backmatter/bibliography:id56}
\sphinxAtStartPar
Paul Hockett. PEMtk \sphinxhyphen{} the Photoelectron Metrology Toolkit \sphinxhyphen{} documentation. 2021. URL: \sphinxurl{https://pemtk.readthedocs.io} (visited on 2022\sphinxhyphen{}02\sphinxhyphen{}18).
\bibitem[5]{backmatter/bibliography:id45}
\sphinxAtStartPar
Paul Hockett. ePSproc: Post\sphinxhyphen{}processing suite for ePolyScat electron\sphinxhyphen{}molecule scattering calculations. \sphinxstyleemphasis{Authorea}, 2016. URL: \sphinxurl{https://www.authorea.com/users/71114/articles/122402/\_show\_article}, \sphinxhref{https://doi.org/10.6084/m9.figshare.3545639}{doi:10.6084/m9.figshare.3545639}.
\bibitem[6]{backmatter/bibliography:id41}
\sphinxAtStartPar
Paul Hockett. ePSproc: Post\sphinxhyphen{}processing for ePolyScat (Github repository). Github, 2016. URL: \sphinxurl{https://github.com/phockett/ePSproc}, \sphinxhref{https://doi.org/10.6084/m9.figshare.3545639}{doi:10.6084/m9.figshare.3545639}.
\bibitem[7]{backmatter/bibliography:id40}
\sphinxAtStartPar
Paul Hockett. ePSproc: Post\sphinxhyphen{}processing for ePolyScat documentation. 2020. URL: \sphinxurl{https://epsproc.readthedocs.io} (visited on 2022\sphinxhyphen{}02\sphinxhyphen{}18).
\bibitem[8]{backmatter/bibliography:id63}
\sphinxAtStartPar
Robert R. Lucchese, Kazuo Takatsuka, and Vincent McKoy. Applications of the Schwinger variational principle to electron\sphinxhyphen{}molecule collisions and molecular photoionization. \sphinxstyleemphasis{Physics Reports}, 131(3):147–221, January 1986. URL: \sphinxurl{http://www.sciencedirect.com/science/article/pii/037015738690147X} (visited on 2012\sphinxhyphen{}07\sphinxhyphen{}18), \sphinxhref{https://doi.org/10.1016/0370-1573(86)90147-X}{doi:10.1016/0370\sphinxhyphen{}1573(86)90147\sphinxhyphen{}X}.
\bibitem[9]{backmatter/bibliography:id43}
\sphinxAtStartPar
F. A. Gianturco, R. R. Lucchese, and N. Sanna. Calculation of low\sphinxhyphen{}energy elastic cross sections for electron\sphinxhyphen{}CF4 scattering. \sphinxstyleemphasis{The Journal of Chemical Physics}, 100(9):6464, May 1994. URL: \sphinxurl{http://scitation.aip.org/content/aip/journal/jcp/100/9/10.1063/1.467237} (visited on 2015\sphinxhyphen{}08\sphinxhyphen{}13), \sphinxhref{https://doi.org/10.1063/1.467237}{doi:10.1063/1.467237}.
\bibitem[10]{backmatter/bibliography:id66}
\sphinxAtStartPar
Alexandra P P Natalense and Robert R Lucchese. Cross section and asymmetry parameter calculation for sulfur 1s photoionization of SF{[}sub 6{]}. \sphinxstyleemphasis{The Journal of Chemical Physics}, 111(12):5344, 1999. URL: \sphinxurl{http://link.aip.org/link/JCPSA6/v111/i12/p5344/s1\&Agg=doi} (visited on 2012\sphinxhyphen{}07\sphinxhyphen{}18), \sphinxhref{https://doi.org/10.1063/1.479794}{doi:10.1063/1.479794}.
\bibitem[11]{backmatter/bibliography:id64}
\sphinxAtStartPar
R R Lucchese. ePolyScat User's Manual. URL: \sphinxurl{https://epolyscat.droppages.com/} (visited on 2022\sphinxhyphen{}04\sphinxhyphen{}26).
\bibitem[12]{backmatter/bibliography:id39}
\sphinxAtStartPar
Andrew C. Brown, Gregory S. J. Armstrong, Jakub Benda, Daniel D. A. Clarke, Jack Wragg, Kathryn R. Hamilton, Zdeněk Mašín, Jimena D. Gorfinkiel, and Hugo W. van der Hart. RMT: R\sphinxhyphen{}matrix with time\sphinxhyphen{}dependence. Solving the semi\sphinxhyphen{}relativistic, time\sphinxhyphen{}dependent Schrödinger equation for general, multielectron atoms and molecules in intense, ultrashort, arbitrarily polarized laser pulses. \sphinxstyleemphasis{Computer Physics Communications}, 250:107062, May 2020. URL: \sphinxurl{https://www.sciencedirect.com/science/article/pii/S0010465519303856} (visited on 2022\sphinxhyphen{}11\sphinxhyphen{}09), \sphinxhref{https://arxiv.org/abs/1905.06156}{arXiv:1905.06156}, \sphinxhref{https://doi.org/10.1016/j.cpc.2019.107062}{doi:10.1016/j.cpc.2019.107062}.
\bibitem[13]{backmatter/bibliography:id68}
\sphinxAtStartPar
Andrew C. Brown, Gregory S. J. Armstrong, Jakub Benda, Daniel D. A. Clarke, Jack Wragg, Kathryn R. Hamilton, Zdeněk Mašín, Jimena D. Gorfinkiel, and Hugo W. van der Hart. RMT: R\sphinxhyphen{}matrix with time\sphinxhyphen{}dependence (repository). May 2020. URL: \sphinxurl{https://gitlab.com/Uk-amor/RMT/rmt} (visited on 2022\sphinxhyphen{}11\sphinxhyphen{}09), \sphinxhref{https://arxiv.org/abs/1905.06156}{arXiv:1905.06156}.
\bibitem[14]{backmatter/bibliography:id42}
\sphinxAtStartPar
Michael W Schmidt, Kim K Baldridge, Jerry A Boatz, Steven T Elbert, Mark S Gordon, Jan H Jensen, Shiro Koseki, Nikita Matsunaga, Kiet A Nguyen, Shujun Su, Theresa L Windus, Michel Dupuis, and John A Montgomery. General atomic and molecular electronic structure system. \sphinxstyleemphasis{Journal of Computational Chemistry}, 14(11):1347–1363, 1993. URL: \sphinxurl{http://dx.doi.org/10.1002/jcc.540141112}, \sphinxhref{https://doi.org/10.1002/jcc.540141112}{doi:10.1002/jcc.540141112}.
\bibitem[15]{backmatter/bibliography:id44}
\sphinxAtStartPar
Mark S. Gordon. Gamess website. URL: \sphinxurl{http://www.msg.ameslab.gov/gamess/}.
\bibitem[16]{backmatter/bibliography:id54}
\sphinxAtStartPar
Paul Hockett. ePS data: Photoionization calculations archive. 2019. URL: \sphinxurl{https://phockett.github.io/ePSdata/} (visited on 2022\sphinxhyphen{}02\sphinxhyphen{}16).
\bibitem[17]{backmatter/bibliography:id55}
\sphinxAtStartPar
Paul Hockett. ePSdata repositories on Zenodo. 2019. URL: \sphinxurl{https://zenodo.org/search?page=1\&size=20\&q=hockett\&keywords=Data}.
\bibitem[18]{backmatter/bibliography:id59}
\sphinxAtStartPar
Paul Hockett. Open Photoionization Docker Stacks. URL: \sphinxurl{https://github.com/phockett/open-photoionization-docker-stacks} (visited on 2022\sphinxhyphen{}08\sphinxhyphen{}04).
\bibitem[19]{backmatter/bibliography:id62}
\sphinxAtStartPar
Marcus Johansson. Libmsym Github. July 2022. URL: \sphinxurl{https://github.com/mcodev31/libmsym} (visited on 2022\sphinxhyphen{}08\sphinxhyphen{}03).
\bibitem[20]{backmatter/bibliography:id70}
\sphinxAtStartPar
Mark A. Wieczorek and Matthias Meschede. SHtools Github. SHTOOLS, August 2022. URL: \sphinxurl{https://github.com/SHTOOLS/SHTOOLS} (visited on 2022\sphinxhyphen{}08\sphinxhyphen{}03).
\bibitem[21]{backmatter/bibliography:id71}
\sphinxAtStartPar
Mark A. Wieczorek and Matthias Meschede. SHTools: Tools for Working with Spherical Harmonics. \sphinxstyleemphasis{Geochemistry, Geophysics, Geosystems}, 19(8):2574–2592, August 2018. URL: \sphinxurl{http://doi.wiley.com/10.1029/2018GC007529} (visited on 2022\sphinxhyphen{}08\sphinxhyphen{}03), \sphinxhref{https://doi.org/10.1029/2018GC007529}{doi:10.1029/2018GC007529}.
\end{sphinxthebibliography}







\renewcommand{\indexname}{Index}
\printindex
\end{document}